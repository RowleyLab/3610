\documentclass[11pt, letterpaper]{memoir}
\usepackage{HomeworkStyle}
\geometry{margin=1in}



\begin{document}

	\begin{center}
		{\large Quiz 2.1 --	Internal energy}
	\end{center}
	{\large Name: \rule[-1mm]{4in}{.1pt} 

\subsection*{Ideal Gas Heat Capacities}
$\circ$ Give the constant volume heat capacities (in the low temperature limit) for perfect gases with the following geometries:
\begin{enumerate}
	\item Monoatomic
	
	\vspace{1em}
	\item Linear Diatomic
	
	\vspace{1em}
	\item Non-linear Polyatomic	
\end{enumerate}


\vspace{1em}
\noindent$\circ$ Describe qualitatively what would happen to these heat capacities in the high temperature limit and why

\vspace{5em}
\noindent $\circ$ Explain why we must designate constant pressure or constant volume for heat capacities

\vspace{5em}
\noindent $\circ$ Predict qualitatively how $C_V$ might compare to $C_p$ for a gas at a given temperature

\vspace{15em}
\subsection*{Work}
One mole of gas at $34^\circ C$ undergoes a reversible isothermal expansion in two stages:
\begin{enumerate}
	\item From $5.0~L$ to $7.5~L$
	\item From $7.5~L$ to $10.0~L$
\end{enumerate}
\noindent $\circ$ Find the work ($w_{sys}$) at each stage

\vspace{3em}
\noindent $\circ$ Explain why the work done is not equal, even though the volume changes are the same

\vspace{4em}
\noindent $\circ$ The gas then undergoes a reversible isothermal compression where $w_{sys}=5500~J$. What is the final volume?

\vspace{4em}
\subsection*{Heat}
$\circ$ $10.0~g$ of \ch{He} gas at $20.0^\circ C$ are heated by $315~J$ at constant volume. What is the final temperature of the gas?

\vspace{4em}
\noindent $\circ$ $10.0~g$ of \ch{N2} gas at $20.0^\circ C$ are heated by $315~J$ at constant volume. What is the final temperature of the gas?

\vspace{4em}
\noindent $\circ$ Find the heat ($q_{sys}$) required to cool $10.0~g$ of methane gas by $5^\circ C$ at constant volume

\newpage
\newgeometry{margin=1.25in}
\pagestyle{empty}
\addtocounter{page}{-1}
\section*{\emph{Птичка (A Little Bird)}}
\paragraph{By Александр Сергеевич Пушкин (Alexander Sergeyevich Pushkin)}~

\begin{verse}
	В чужбине свято наблюдаю\\
	Родной обычай старины:\\
	На волю птичку выпускаю\\
	При светлом празднике весны. 
	
	Я стал доступен утешенью;\\
	За что на бога мне роптать,\\
	Когда хоть одному творенью\\
	Я мог свободу даровать! 
\end{verse}

\vspace{2em}
\begin{verse}
	In alien lands I keep the body\\
	Of ancient native rites and things:\\
	I gladly free a little birdie\\
	At celebration of the spring.
	
	
	I'm now free for consolation,\\
	And thankful to almighty Lord:\\
	At least, to one of his creations\\
	I've given freedom in this world!
\end{verse}
\end{document}
