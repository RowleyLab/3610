\documentclass[11pt, letterpaper]{memoir}
\usepackage{HomeworkStyle}
\usepackage{multirow}
\geometry{margin=1in}



\begin{document}

	\begin{center}
		{\large Quiz 3.2 -- Spontaneity}
	\end{center}
	{\large Name: \rule[-1mm]{4in}{.1pt} 

\subsection*{A Spontaneous Process}

Consider the reaction: \ch{2 H2(g) + O2(g) -> 2 H2O(g)}

\noindent
At $298~K$, this reaction has $\Delta H^{\std} = -483.64\dfrac{kJ}{mol}$ and $\Delta S^{\std}= -88.846\dfrac{J}{mol~K}$. Use these data to find $\Delta S_{sys}$, $\Delta S_{surr}$, and $\Delta S_{univ}$ for this reaction, and show that it is spontaneous under standard conditions

\vspace{4em}
\subsection*{Entropy on Astronomical Scales}
The sun radiates heat from its surface at a temperature of $5,778~K$. Some of this heat reaches the earth, where it is absorbed, and re-emitted at the average global temperature of about $15 ^\circ C$. The Earth's heat radiates away into space, which acts as a heat sink at $2.7~K$ (The temperature of the cosmic microwave background). The Earth's temperature is fairly constant, so we can assume the heat absorbed from the sun and the heat emitted into space are balanced. Find $\Delta S_{Sun}$, $\Delta S_{Earth}$, $\Delta S_{Space}$, and $\Delta S_{Universe}$ for each $J$ of energy along this Sun-Earth-Space radiative pathway.


\vspace{10em}
\subsection*{Conditions for spontaneity}
In class we saw how $\Delta H$ and $\Delta S$ can determine the conditions under which a reaction will be spontaneous. However, that discussion only applied to isobaric processes. Fill out the table below to indicate the conditions under which isochoric processes are spontaneous

\begin{center}
	\begin{tabular}{cc|c|c|}
		&&\multicolumn{2}{c|}{$\Delta S$} \\
		&& $+$ & $-$ \\ \cline{1-4}
		\multirow{2}*{$\Delta U$} &$+$&~\hspace{12em}~& ~\hspace{12em}~\\  \cline{2-4}
		&$-$&& \\ \cline{1-4}
		
	\end{tabular}
\end{center}
\newpage
\subsection*{A Reversible Process}

An isobaric phase change at the phase change temperature is a great simple example of a reversible process. A pot filled with $400.0~g$ of water are boiled on a stove. In this scenario, we can consider the surroundings to be at the boiling temperature.

\noindent Find $\Delta S_{sys}$, $\Delta S_{surr}$, and $\Delta S_{univ}$ for this process

\vspace{6em}
\subsection*{A Non-Spontaneous Process. . . which isn't (or, rather, which really \emph{is} spontaneous)}

A pot filled with $400.0~g$ left out $298~K$ may evaporate into the vapor phase given enough time. Find $\Delta S_{sys}$, $\Delta S_{surr}$, and $\Delta S_{univ}$ for this phase change and show that it is non-spontaneous. (Remember that $\Delta H$ must be corrected by Kirchoff's law)

\vspace{15em}\noindent
And yet, you \emph{know} that such a process will happen spontaneously in nature. . . at least, in Utah it will. What factor have we neglected, and how can we correct our calculations to include it?


\newpage
\pagestyle{empty}
\addtocounter{page}{-1}
\section*{\emph{{\fontspec{Microsoft JhengHei}七步詩} (The Quatrain of Seven Steps)}}
\paragraph{By {\fontspec{Microsoft JhengHei}曹植} (Cao Zhi)}~

{\fontspec{Microsoft JhengHei}
	\begin{verse}
		煮豆燃豆萁\\
		漉菽以為汁\\
		萁在釜下燃\\
		豆在釜中泣\\
		本是同根生\\
		相煎何太急
	\end{verse}
}

\vspace{2em}
\begin{verse}
	People burn the beanstalk to boil beans,\\
	filtering them to extract juice.\\
	The beanstalks were burnt under the cauldron,\\
	and the beans in the cauldron wailed:\\
	“We were originally grown from the same root;\\
	Why should we hound each other to death with such impatience?”
\end{verse}
\end{document}
