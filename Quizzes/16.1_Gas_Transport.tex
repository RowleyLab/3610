\documentclass[11pt, letterpaper]{memoir}
\usepackage{HomeworkStyle}
\geometry{margin=1in}



\begin{document}

	\begin{center}
		{\large Quiz 16.1 -- Transport in Gases}
	\end{center}
	{\large Name: \rule[-1mm]{4in}{.1pt} 


\subsection*{The Diffusion Coefficient}
Find the diffusion coefficients for both \ch{He} gas and \ch{N2} at $25^\circ C$ and $0.82~atm$ (Typical values for\\Cedar City)

\vspace{18em}\noindent
The Earth's atmosphere grows thinner at higher altitudes with a concentration gradient of approximately $\frac{\mathrm{d}\left[\ch{N2}\right]}{\mathrm{d}z}\approx-8\times10^{-7}\frac{M}{m}$. This gradient can be used in Fick's first law just like $\frac{\mathrm{d}\mathcal{N}}{\mathrm{d}z}$, and will just give a flux in units of molar concentration rather than units of number density. The atmosphere (thankfully) doesn't diffuse away into space because of the pull of gravity. If gravity suddenly stopped functioning, what flux ($J_{Matter}$) we would expect as the atmosphere begins its escape from Cedar City into space?

\vspace{20em}
\subsection*{The Other Transport Coefficients}
Consider a gas with $D=1.5\times10^{-5}\frac{m^2}{s}$ at $15^\circ C$ and $1.2~atm$

\noindent Find the coefficient of thermal conductivity ($\kappa$) for this gas if it is:
\begin{itemize}
	\item A noble gas:
	
	\vspace{6em}
	\item A non-linear polyatomic gas:
\end{itemize}

\vspace{6em}\noindent
In each case, find the flux of energy if the gas has a temperature gradient of $\frac{\mathrm{d}T}{\mathrm{d}z}=5\frac{K}{m}$

\vspace{11em}\noindent
Find the coefficient of viscosity for the gas if has a molar mass of:
\begin{itemize}
	\item $37\frac{g}{mol}$
	
	\vspace{6em}
	\item $115\frac{g}{mol}$
\end{itemize}

\newpage
\pagestyle{empty}
\addtocounter{page}{-1}	
\section*{\emph{{\fontspec{Malgun Gothic}오늘} (Today)}}
\paragraph{By {\fontspec{Malgun Gothic}구상} (Ku Sang)}~

{\fontspec{Malgun Gothic}
	\begin{verse}
		오늘도 신비의 샘인 하루를 맞는다.
		
		이 하루는 저 강물의 한 방울이\\
		어느 산골짝 옹달샘에 이어져 있고\\
		아득한 푸른 바다에 이어져 있듯\\
		과거와 미래와 현재가 하나다.
		
		이렇듯 나의 오늘은 영원 속에 이어져\\
		바로 시방 나는 그 영원을 살고 있다.
		
		그래서 나는 죽고 나서부터가 아니라\\
		오늘서부터 영원을 살아야 하고\\
		영원에 합당한 삶을 살아야 한다.
		
		마음이 가난한 삶을 살아야 한다.\\
		마음을 비운 삶을 살아야 한다.
	\end{verse}
}

\vspace{2em}
\begin{verse}
	Today again I meet a day, a well of mystery.
	
	Like a drop of that river extends to\\
	a spring of a valley and then to\\
	the faraway blue sea, for this day\\
	the past, the future, and the present are one.
	
	So does my today extend to eternity,\\
	and right now I am living the eternity.
	
	So, starting from today, I should live\\
	eternity, not after I die,\\
	and should live a life that deserves eternity.
	
	I should live the life of a poor heart.\\
	I should live the life of an empty heart.
\end{verse}


\end{document}
