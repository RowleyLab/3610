\documentclass[11pt, letterpaper]{memoir}
\usepackage{HomeworkStyle}
\geometry{margin=1in}



\begin{document}

	\begin{center}
		{\large Quiz 2.2 --	Enthalpy}
	\end{center}
	{\large Name: \rule[-1mm]{4in}{.1pt} 

\subsection*{Ideal Gas Heat Capacities}
\noindent$\circ$ Give the constant pressure heat capacities (in the low temperature limit) for perfect gases with the following geometries:
\begin{enumerate}
	\item Monoatomic
	
	\vspace{1em}
	\item Linear Diatomic
	
	\vspace{1em}
	\item Non-linear Polyatomic	
\end{enumerate}

\vspace{1em}
\subsection*{Definition of Enthalpy}
$\circ$ What is the mathematical equation which relates enthalpy to internal energy?

\vspace{3em}
\noindent$\circ$ Under what conditions is $\Delta_{rxn}U\approx \Delta_{rxn}H$

\vspace{2em}
\subsection*{Pressure-Volume Work}
$\circ$ A gas expands under isobaric conditions at $0.75~atm$ from $0.50~L$ to $2.75~L$. What is the work done (from the perspective of the system)?

\vspace{3em}
\noindent $\circ$ A pressurized tank with $V=15.0~L$ contains a gas at $140.0~atm$. The tank is slowly leaking its gas into the atmosphere with a barometric pressure of $0.82~atm$. The leak is slow enough that the temperature remains constant throughout. What is the total work ($w_{sys}$) for the gas as it leaks? (Note that although this process is isothermal, it is not reversible since $p\neq p_{ext}$)

\vspace{4em}
\noindent $\circ$ A similar tank to the one above is mounted on the exterior of the international space station. This tank also experiences a leak like the one above. What is the total work ($w_{sys}$) for the gas as it leaks? (Unlike the example above, this process cannot be isothermal but temperature isn't important to this problem anyway)
\newpage
\pagestyle{empty}
\addtocounter{page}{-1}
\newgeometry{vmargin=0.9in, hmargin=1.25in}
\section*{\emph{Caged Bird}}
\paragraph{By Maya Angelou}~
\begin{verse}
	A free bird leaps\\
	on the back of the wind\\
	and floats downstream\\ 
	till the current ends\\
	and dips his wing\\
	in the orange sun rays\\
	and dares to claim the sky.
	
	But a bird that stalks\\
	down his narrow cage\\
	can seldom see through\\
	his bars of rage\\
	his wings are clipped and\\
	his feet are tied\\
	so he opens his throat to sing.
	
	The caged bird sings\\
	with a fearful trill\\ 
	of things unknown\\
	but longed for still\\
	and his tune is heard\\
	on the distant hill\\
	for the caged bird\\
	sings of freedom.
	
	The free bird thinks of another breeze\\
	and the trade winds soft through the sighing trees\\
	and the fat worms waiting on a dawn bright lawn\\
	and he names the sky his own
	
	But a caged bird stands on the grave of dreams\\
	his shadow shouts on a nightmare scream\\
	his wings are clipped and his feet are tied\\
	so he opens his throat to sing.\\
	
	The caged bird sings\\
	with a fearful trill\\ 
	of things unknown\\
	but longed for still\\
	and his tune is heard\\
	on the distant hill\\
	for the caged bird\\
	sings of freedom.\\	
\end{verse}
\end{document}
