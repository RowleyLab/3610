\documentclass[11pt, letterpaper]{memoir}
\usepackage{HomeworkStyle}
\geometry{margin=1in}



\begin{document}

	\begin{center}
		{\large Quiz 3.1 -- Entropy}
	\end{center}
	{\large Name: \rule[-1mm]{4in}{.1pt} 
		
\subsection*{Carnot cycle}
Consider a heat engine based around the Carnot cycle. Sketch this cycle on a p/V diagram, labeling the states in the process as A, B, C, and D

\vspace{6em}\noindent
Tell which direction around this cycle operates as a heat engine, and which direction operates as a heat pump
\begin{itemize}
	\item A-B-C-D:
	\item A-D-C-B: 
\end{itemize}

\noindent
Fill in the table below for the cycle when operating as a heat engine. Use generic variables ($C_V$, $T_H$, $T_C$, $V_A$, $V_B$, etc.)

\noindent
\begin{tabular}{|c|c|c|c|c|}
	\toprule
	Step & $w$ & $q$ & $\Delta U$ & $\Delta S$ \\ \midrule
	A & \hspace{1.25in} & \hspace{1.25in} & \hspace{1.25in} & \hspace{1.25in} \\ \midrule
	B &&&&\\ \midrule
	C &&&&\\ \midrule
	D &&&&\\ \midrule \midrule
	net (A-D) &&&& \\\bottomrule
\end{tabular}

\vspace{1em}
\noindent
A car engine is a type of heat engine, and burns gasoline burns at about $600~^\circ C$. If the ambient temperature is $25~^\circ C$, what is the thermodynamic maximum efficiency a car engine can achieve?

\vspace{6em}
\subsection*{Measuring molar entropy}
He has $T_{boil}=4.25~K$ and $\Delta H_{vap} = 83\frac{J}{mol}$. The isobaric heat capacity for liquid helium is very complex, but can be approximated as $C_p(l)\approx 7.4\times10^{-3}T^3\frac{J}{mol~K}$. The isobaric heat capacity for gaseous He is simply $C_p(g)=\frac{5}{2}R$. Use these data to calculate the molar entropy for He gas at room temperature, and compare it to the value given in our textbook appendix.

\vspace{6em}
\subsection*{Irreversibility in Mechanical Systems}
Consider a spring which obeys Hook's law: $F=-kx$ where $x$ is the displacement away from equilibrium and $k=650\dfrac{N}{m}$. The acceleration due to gravity is $9.80665~\dfrac{m}{s^2}$.

$\circ$ Calculate the equilibrium displacement if a $10~kg$ weight is placed on the spring

\noindent 

\vspace{4em}
\noindent
Considering the same weight-on-a-spring in Problem 1:

$\circ$ Calculate the work done by the falling weight.

\vspace{5em}
\noindent
How much work would be done if instead the spring was stretched reversibly to the same equilibrium displacement. Bonus -- Explain the discrepancy!

\vspace{5em}
\noindent
The spring-weight system will lose kinetic energy through friction with the air until it rests at its equilibrium position. What is $\Delta S_{universe}$ for both the reversible and irreversible processes if they are done at room temperature ($25^\circ C$)?

\newpage
\pagestyle{empty}
\addtocounter{page}{-1}
\section*{\emph{Who Has Seen the Wind?}}
\paragraph{By Christina Rossetti}~
\begin{verse}
	Who has seen the wind?\\
	Neither I nor you:\\
	But when the leaves hang trembling,\\
	The wind is passing through.
	
	Who has seen the wind?\\
	Neither you nor I:\\
	But when the trees bow down their heads,\\
	The wind is passing by.
\end{verse}
\end{document}
