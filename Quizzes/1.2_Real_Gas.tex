\documentclass[11pt, letterpaper]{memoir}
\usepackage{HomeworkStyle}
\geometry{margin=1in}



\begin{document}

	\begin{center}
		{\large Quiz 1.2 --	Real Gases}
	\end{center}
	{\large Name: \rule[-1mm]{4in}{.1pt} 

\subsection*{Question 1}
Using the table of Van der Waals constants at the end of your textbook, find the following (From those listed in the table):
\begin{itemize}
	\item Smallest gas particle
	
	\vspace{1em}
	\item Largest gas particle
	
	\vspace{1em}
	\item Gas with the strongest attractive forces
	
	\vspace{1em}
	\item Gas with the weakest attractive forces
\end{itemize}

\vspace{1em}
\subsection*{Question 2}
Ammonia has Van der Waals constants of $a=4.225\dfrac{L^2bar}{mol^2}$ and $b=0.0371\dfrac{L}{mol}$

$10.0~mol$ of ammonia are placed in $0.450~L$ at $315~K$. Find the following:
\begin{itemize}
	\item Pressure (bar) assuming ideal behavior
	
	\vspace{3em}
	\item Pressure (bar) using the Van der Waals equation
	
	\vspace{4em}
	\item Compression factor ($Z$)
	
	\vspace{4em}
	\item Reduced state variables, $V_r$, $p_r$, and $T_r$ (You will need to refer to your textbook)
\end{itemize}

\vspace{6em}
\subsection*{Question 3}
Explain the significance of the Boyle temperature of a gas as it relates to:
\begin{itemize}
	\item The virial equation --
	
	\vspace{4em}
	\item The ideal gas law --
\end{itemize}

\vspace{4em}
\subsection*{Question 4}


\newpage
\newgeometry{margin=1.25in}
\pagestyle{empty}
\addtocounter{page}{-1}
\section*{\emph{The Waves}}
\paragraph{By Virginia Woolf}~

\vspace{1em}
\begin{minipage}{0.3\textwidth}
	\begin{center}
		I see nothing.
		
		We may sink and settle\\
		on the waves.\\
		The sea will drum\\
		in my ears.
		
		The white petals\\
		will be darkened\\
		with sea water.
		
		They will float\\
		for a moment\\
		and then sink.
		
		Rolling over\\
		the waves will\\
		shoulder me under.
		
		Everything falls in a\\
		tremendous shower,\\
		dissolving me.
	\end{center}
\end{minipage}
\end{document}
