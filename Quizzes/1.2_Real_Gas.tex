\documentclass[11pt, letterpaper]{memoir}
\usepackage{HomeworkStyle}
\geometry{margin=1in}



\begin{document}

	\begin{center}
		{\large Quiz 1.2 --	Real Gases}
	\end{center}
	{\large Name: \rule[-1mm]{4in}{.1pt} 

\subsection*{Question 1}
Using Table 1C.3 (van der Waals coefficients) at the end of your textbook, find the following (From those listed in the table):
\begin{itemize}
	\item Smallest gas particle
	
	\vspace{1em}
	\item Largest gas particle
	
	\vspace{1em}
	\item Gas with the strongest attractive forces
	
	\vspace{1em}
	\item Gas with the weakest attractive forces
\end{itemize}

\vspace{1em}
\subsection*{Question 2}
Ammonia has van der Waals constants of $a=4.169\dfrac{L^2bar}{mol^2}$ and $b=0.0371\dfrac{L}{mol}$

\noindent $8.00~mol$ of ammonia are placed in $2.75~L$ at $348~K$. Find the following:
\begin{itemize}
	\item Pressure (bar) assuming ideal behavior
	
	\vspace{3em}
	\item Pressure (bar) using the van der Waals equation
	
	\vspace{4em}
	\item Compression factor ($Z$) using this van der Waals pressure
	
	\vspace{4em}
	\item Reduced state variables, $V_r$, $p_r$, and $T_r$ (You will need to refer to your textbook)
\end{itemize}

\vspace{6em}
\subsection*{Question 3}
Explain the significance of the Boyle temperature of a gas as it relates to:
\begin{itemize}
	\item The virial equation --
	
	\vspace{4em}
	\item The ideal gas law --
\end{itemize}

\vspace{4em}
\subsection*{Question 4}
Ammonia has second virial coefficient of $B=-165\dfrac{cm^3}{mol}$ at $348~K$

\noindent $8.00~mol$ of ammonia are placed in $2.75~L$ at $348~K$ 
\begin{itemize}
	\item Find the pressure (bar) using the virial equation
	
	\vspace{4em}
	\item Find the compression factor ($Z$) using this virial pressure
	
	\vspace{4em}
	\item Compare the van der Waals pressure calculated above to this virial pressure
	
	\vspace{4em}
	\item Explain how the results of the virial equation can rival those of the van der Waals equation, when it uses only one corrective term and the van der Waals equation uses two
\end{itemize}

\newpage
\newgeometry{margin=1.25in}
\pagestyle{empty}
\addtocounter{page}{-1}
\section*{\emph{The Waves}}
\paragraph{By Virginia Woolf}~

\vspace{1em}
\begin{minipage}{0.3\textwidth}
	\begin{center}
		I see nothing.
		
		We may sink and settle\\
		on the waves.\\
		The sea will drum\\
		in my ears.
		
		The white petals\\
		will be darkened\\
		with sea water.
		
		They will float\\
		for a moment\\
		and then sink.
		
		Rolling over\\
		the waves will\\
		shoulder me under.
		
		Everything falls in a\\
		tremendous shower,\\
		dissolving me.
	\end{center}
\end{minipage}
\end{document}
