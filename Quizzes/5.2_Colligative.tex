\documentclass[11pt, letterpaper]{memoir}
\usepackage{HomeworkStyle}
\geometry{margin=1in}



\begin{document}

	\begin{center}
		{\large Quiz 5.2 -- Colligative Properties}
	\end{center}
	{\large Name: \rule[-1mm]{4in}{.1pt} 
		
\subsection*{Phase Change Temperatures}
Our textbook gives the following formula for the boiling point elevation constant $K_b=\dfrac{RT^{\star 2}}{\Delta H_{vap}}$
\\ Find $K_b$ for water, using $\Delta H_{vap} = 40.7\dfrac{kJ}{mol}$

\vspace{4em}
\noindent Our textbook also provides some values of $K_b$ in Table 5B.1, including for water: $K_b=0.51~\dfrac{K~kg}{mol}$
\\ Show how these two values are actually consistent with each other

\vspace{4em}
\noindent Benzene has $K_f=-5.12\dfrac{K~kg}{mol}$ and a normal freezing point of 5.5°C. If 1.6 g of naphthalene are dissolved into 5.6 g of benzene, what is the new freezing temperature?

\vspace{4em}
\subsection*{Osmotic Pressure}
Seawater contains about 35 g of \ch{NaCl} in every kg of water solvent. Seawater can be purified through reverse osmosis, but requires applying a pressure equal to the osmotic pressure. What is the osmotic pressure of seawater at 25°C?

\vspace{4em}
\noindent2.5 g of an unknown non-electrolyte are dissolved in water to make 100.0 ml of solution. At 25°C the solution exhibits an osmotic pressure of 1.79 atm. What is the molar mass of the unknown?

\vspace{4em}
\noindent What would the molar mass be if the unknown compound were instead a salt of the form \ch{A2B3}?

\newpage
\pagestyle{empty}
\addtocounter{page}{-1}
\section*{\emph{Элегия (Elegy)}}
\paragraph{By Александр Сергеевич Пушкин (Alexander Sergeyevich Pushkin)}~
\begin{verse}
	Безумных лет угасшее веселье\\
	Мне тяжело, как смутное похмелье.\\
	Но, как вино — печаль минувших дней\\
	В моей душе чем старе, тем сильней.\\
	Мой путь уныл. Сулит мне труд и горе\\
	Грядущего волнуемое море.
	
	Но не хочу, о други, умирать;\\
	Я жить хочу, чтоб мыслить и страдать;\\
	И ведаю, мне будут наслажденья\\
	Меж горестей, забот и треволненья:\\
	Порой опять гармонией упьюсь,\\
	Над вымыслом слезами обольюсь,\\
	И может быть — на мой закат печальный\\
	Блеснет любовь улыбкою прощальной.
\end{verse}

\vspace{2em}
\begin{verse}
	The vanished joy of my crazy years\\
	Is as heavy as gloomy hang-over.\\
	But, like wine, the sorrow of past days\\
	Is stronger with time.\\
	My path is sad. The waving sea of the future\\
	Promises me only toil and sorrow.
	
	But, O my friends, I do not wish to die,\\
	I want to live – to think and suffer.\\
	I know, I’ll have some pleasures\\
	Among woes, cares and troubles.\\
	Sometimes I’ll be drunk with harmony again,\\
	Or will weep over my visions,\\
	And it’s possible, at my sorrowful decline,\\
	Love will flash with a parting smile
\end{verse}
\end{document}
