\documentclass[11pt, letterpaper]{memoir}
\usepackage{HomeworkStyle}
\geometry{margin=1in}



\begin{document}

	\begin{center}
		{\large Quiz 5.2 -- Colligative Properties}
	\end{center}
	{\large Name: \rule[-1mm]{4in}{.1pt} 
		


\newpage
\pagestyle{empty}
\addtocounter{page}{-1}
\section*{\emph{Элегия (Elegy)}}
\paragraph{By Александр Сергеевич Пушкин (Alexander Sergeyevich Pushkin)}~
\begin{verse}
	Безумных лет угасшее веселье\\
	Мне тяжело, как смутное похмелье.\\
	Но, как вино — печаль минувших дней\\
	В моей душе чем старе, тем сильней.\\
	Мой путь уныл. Сулит мне труд и горе\\
	Грядущего волнуемое море.
	
	Но не хочу, о други, умирать;\\
	Я жить хочу, чтоб мыслить и страдать;\\
	И ведаю, мне будут наслажденья\\
	Меж горестей, забот и треволненья:\\
	Порой опять гармонией упьюсь,\\
	Над вымыслом слезами обольюсь,\\
	И может быть — на мой закат печальный\\
	Блеснет любовь улыбкою прощальной.
\end{verse}

\vspace{2em}
\begin{verse}
	The vanished joy of my crazy years\\
	Is as heavy as gloomy hang-over.\\
	But, like wine, the sorrow of past days\\
	Is stronger with time.\\
	My path is sad. The waving sea of the future\\
	Promises me only toil and sorrow.
	
	But, O my friends, I do not wish to die,\\
	I want to live – to think and suffer.\\
	I know, I’ll have some pleasures\\
	Among woes, cares and troubles.\\
	Sometimes I’ll be drunk with harmony again,\\
	Or will weep over my visions,\\
	And it’s possible, at my sorrowful decline,\\
	Love will flash with a parting smile
\end{verse}
\end{document}
