\documentclass[12pt, openany, letterpaper]{memoir}
\usepackage{HomeworkStyle}

\begin{document}

\begin{center}
	{\large Homework 16 -- Molecules in Motion}
\end{center}

Name: \rule[-.1mm]{15em}{0.1pt}

\begin{description}
	\item [Exercise 16A.2(a)] ~ (10 points)

	      Calculate the diffusion constant of argon at $20~^\circ C$ and (i) $1.00~Pa$, (ii) $100.0~kPa$, and (iii) $10.0~MPa$. If a pressure gradient of $1.0~\nicefrac{bar}{m}$ is established in a pipe, what is the flow of gas due to diffusion?


	      \vspace{19em}
	\item [Exercise 16A.6(a)] ~ (5 points)

	      Use the experimental value of the coefficient of viscosity for neon (Table 16A.1) to estimate the collision cross-section of \ch{Ne} atoms at $273~K$.

	      \vspace{20em}
	\item [Exercise 16B.1(a)] ~ (5 points)
	      The viscosity of water at $20~^\circ C$ is $1.002~cP$ and $0.7975~cP$ at $30~^\circ C$. What is the energy of activation for the transport process?

	      \vspace{23em}
	\item [Exercise 16B.3(a)] ~ (5 points)

	      The mobility of a \ch{Rb^{+}} ion in aqueous solution is $7.92\times10^{-8}\nicefrac{m^2}{Vs}$ at $25~^\circ C$. The potential difference between two electrodes placed in the solution is $25.0~V$. If the electrodes are $7.00~mm$ apart, what is the drift speed of the \ch{Rb^{2+}} ion?

	      \vspace{23em}
	\item [Exercise 16B.6(a)] ~ (5 points)

	      Estimate the effective radius of a sucrose molecule in water at $25~^\circ C$ given that its diffusion coefficient is $5.2\times10^{-10}\nicefrac{m^2}{s}$ and the viscosity of water is $1.00~cP$

	      \vspace{23em}
	\item [Exercise 16C.5(a)] ~ (5 points)

	      The diffusion coefficient of \ch{CCl4} in heptane at $25~^\circ C$ is $3.17\times10^{-9}~\nicefrac{m^2}{s}$. Estimate the time required for a \ch{CCl4} molecule to have a root-mean-square displacement of $5.0~mm$.

	      \vspace{23em}
	\item [Exercise 16C.2(a)] ~ (10 points)

	      A layer of $20.0~g$ of sucrose is spread uniformly over a surface of area $5.0~cm^2$ and covered in water to a depth of $20~cm$. What will be the molar concentration of sucrose molecules at $10~cm$ above the original layer at (i) $10~s$, and (ii) $24~h$? Assume diffusion is the only transport process and take $D=5.216\times10^{-9}~\nicefrac{m^2}{s}$.
\end{description}
\newpage
\pagestyle{empty}
\addtocounter{page}{-1}
\section*{\emph{On Shakespeare. 1630}}
\paragraph{By John Milton}~
\begin{verse}
	What needs my Shakespeare for his honoured bones,\\
	The labor of an age in pilèd stones,\\
	Or that his hallowed relics should be hid\\
	Under a star-ypointing pyramid?\\
	Dear son of Memory, great heir of fame,\\
	What need’st thou such weak witness of thy name?\\
	Thou in our wonder and astonishment\\
	Hast built thyself a live-long monument.\\
	For whilst to th’ shame of slow-endeavouring art,\\
	Thy easy numbers flow, and that each heart\\
	Hath from the leaves of thy unvalued book\\
	Those Delphic lines with deep impression took,\\
	Then thou, our fancy of itself bereaving,\\
	Dost make us marble with too much conceiving;\\
	And so sepúlchred in such pomp dost lie,\\
	That kings for such a tomb would wish to die.
\end{verse}
\end{document}
