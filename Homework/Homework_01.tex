\documentclass[12pt, openany, letterpaper]{memoir}
\usepackage{HomeworkStyle}

\begin{document}

\begin{center}
	{\large Homework 1 -- Gas Properties}
	
\end{center}

Name: \rule[-.1mm]{15em}{0.1pt}

\begin{description}
	\item [Excercise 1A.3] ~ (10 points)
	\begin{description}
		\item [~(a)~~] A perfect gas undergoes isothermal compression which reduces its volume by $2.20~dm^3$. The final pressure and volume of the gas are $5.04~bar$ and $4.65~dm^3$, respectively. Calculate the original pressure of the gas in (i) bar, and (ii) atm.
		
		\vspace{6em}
		\item [~(b)~~] A perfect gas undergoes isothermal compression which reduces its volume by $1.80~dm^3$. The final pressure and volume of the gas are $1.97~bar$ and $2.14~dm^3$, respectively. Calculate the original pressure of the gas in (i) bar, and (ii) atm.
				
		\vspace{6em}
	\end{description} 
	\item [Exercise 1A.9] ~ (10 points)
	\begin{description}
		\item [~(a)~~] The density of a gaseous compound was found to be $1.23~\nicefrac{kg}{m^3}$ at $330~K$ and $20~kPa$. What is the molar mass of the compound?
		
		\vspace{9em}
		\item [~(b)~~] In an experiment to measure the molar mass of a gas, $250~cm^3$ of the gas was confined in a glass vessel. The pressure was $152~Torr$ at $298~K$, and after correcting for buoyancy effects, the mass of the gas was $33.5~mg$. What is the molar mass of the gas?
				
		\vspace{9em}
	\end{description} 
	\item [Exercise 1B.7] ~ (10 points)
	\begin{description}
		\item [~(a)~~] Assume that air consists of \ch{N2} molecules with a collision diameter of $395~pm$. Calculate (i) The mean speed of the molecules, (ii) the mean free path, and (iii) the collision frequency in air at $1.0~atm$ and $25^\circ C$.
		
		\vspace{12em}
		\item [~(b)~~] The best laboratory vacuum pump can generate a vacuum of about $1.0~nTorr$. At $25^\circ C$, and assuming that air consists of \ch{N2} molecules with a collision diameter of $395~pm$, calculate (i) The mean speed of the molecules, (ii) the mean free path, and (iii) the collision frequency
				
		\vspace{12em}
	\end{description}
	\item [Excercise 1C.9] ~ (10 points)
	\begin{description}
		\item [~(a)~~] A certain gas obeys the van der Waals equation with $a=0.50~\nicefrac{m^6Pa}{mol^2}$. Its volume is found to be $5.00\times10^{-4}\nicefrac{m^3}{mol}$ at $273~K$ and $3.0~MPa$. From this information calculate the van der Waals constant $b$. What is the compression factor for this gas at the prevailing temperature and pressure? 
		
		\vspace{12em}
		\item [~(b)~~] A certain gas obeys the van der Waals equation with $a=0.76~\nicefrac{m^6Pa}{mol^2}$. Its volume is found to be $4.00\times10^{-4}\nicefrac{m^3}{mol}$ at $288~K$ and $4.0~MPa$. From this information calculate the van der Waals constant $b$. What is the compression factor for this gas at the prevailing temperature and pressure? 
				
		\vspace{12em}
	\end{description} 
	\item [Problem 1C.3] ~ (5 points)
	
	At $273~K$, measurements on argon gave $B=-21.7~\nicefrac{cm^3}{mol}$ and $C=1200~\nicefrac{cm^6}{mol^2}$, where $B$ and $C$ are the second and third virial coefficients in the expansion of $Z$ in powers of $\dfrac{1}{V_m}$. Assuming that the perfect gas law holds sufficiently well for the estimation of the second and third terms of the expansion, calculate the compression factor of argon at $100~atm$ and $273~K$. From your result, estimate the molar volume of argon under these conditions.
	
	\vspace{12em}
	\item [Discussion Question 1C.1] ~ (5 points)
	
	Explain how the compression factor varies with pressure and temperature and describe how it reveals information about intermolecular interactions in real gases.
\end{description}
\end{document}