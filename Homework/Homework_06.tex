\documentclass[12pt, openany, letterpaper]{memoir}
\usepackage{HomeworkStyle}

\begin{document}

\begin{center}
	{\large Homework 6 -- Chemical Equilibrium}
	
\end{center}

Name: \rule[-.1mm]{15em}{0.1pt}

\begin{description}	
	\item [Exercise 6A.7(a)] ~ (5 points)
	
	Establish the relation between $K$ and $K_C$ for the reaction: \ch{H2CO(g) <-> CO(g) + H2(g)}
	
	\vspace{18em}
	\item [Exercise 6A.10(a)] ~ (5 points)
	
	The standard Gibbs energy of formation of \ch{NH3(g)} is $-16.5~\nicefrac{kJ}{mol}$ at $298~K$. What is the reaction Gibbs energy when the partial pressures of the \ch{N2}, \ch{H2}, and \ch{NH3} (treated as perfect gases) are $3.0~bar$, $1.0~bar$, and $4.0~bar$, respectively? What is the spontaneous direction of the reaction in this case?
	
	\vspace{20em}
	\item [Exercise 6B.1(a)] ~ (5 points)
	
	The standard reaction enthalpy of \ch{Zn(s) + H2O(g) <-> ZnO(s) + H2(g)} is approximately constant at $+224~\nicefrac{kJ}{mol}$ from $920~K$ up to $1280~K$. The standard reaction Gibbs energy is $+33~\nicefrac{kJ}{mol}$ at $1280~K$. Estimate the temperature at which the equilibrium constant becomes greater than 1.
	
	\vspace{20em}
	\item [My Problem 1] ~ (10 points)
	
	Syngas, a mixture of carbon monoxide and hydrogen gas, can be produced by reacting methane with water in the reaction below:
	
	\ch{CH4(g) + H2O(g) <-> CO(g) + 3 H2(g)} \hspace{2em} $\Delta_{rxn}H^{\std}(273~K)=206.13~\nicefrac{kJ}{mol}$
	
	\noindent You are a chemical engineer designing a new syngas production plant, and you want to maximize the amount of syngas produced (maximize $\xi$) at equilibrium.$^*$
	
	\noindent $\circ$ Should you run the reaction at high or low temperature?
	
	\vspace{7em}
	\noindent $\circ$ Should you run the reaction with high partial pressures or low partial pressures?

	\vspace{7em}
	*Really, chemical engineers consider far more diverse and complex factors when designing plants
	
	\vspace{2em}
	\item [Exercise 6C.2(a)] ~ (15 points)
	
	Devise cells in which the following are the reactions and calculate the standard cell potential in each case. Write your devised cell using line notation (e.g. \ch{Cd|Cd(NO3)2(aq)||NiCl2(aq)|Ni})
	\begin{enumerate}
		\item \ch{Zn(s) + CuSO4(aq) -> ZnSO4(aq) + Cu(s)}
		
		\vspace{6em}
		\item \ch{2 AgCl(s) + H2(g) -> 2 HCl(aq) + a Ag(s)}
		
		\vspace{6em}
		\item \ch{2 H2(g) + O2(g) -> 2 H2O(l)}
	\end{enumerate}
	
	\vspace{6em}
	
	\item [Exercise 6C.3(a)] ~ (5 points)
	
	Use the Debye-H\"uckel limiting law and the Nernst equation to estimate the potential of the cell \ch{Ag|AgBr(s)|KBr(aq,~ $0.050~\nicefrac{mol}{kg}$ \!)||Cd(NO3)2(aq,~ $0.010~\nicefrac{mol}{kg}$ \!)|Cd} at $25^\circ C$.
	
	\vspace{25em}
	\item [Exercise 6D.1(a)] ~ (10 points)
	
	Calculate equilibrium constants of the following reactions at $25^\circ C$ from standard potential data:
	\begin{enumerate}
		\item \ch{Sn(s) + Sn^{4+}(aq) <-> 2 Sn^{2+}(aq)}
		
		\vspace{13em}
		\item \ch{Sn(s) + 2 AgCl(s) <-> SnCl2(aq) + 2 Ag(s)}
	\end{enumerate}

	\vspace{13em}
	\item [My Problem 2] ~ (5 Points)
	
	Find the standard reduction potential for \ch{Cu^{2+} + 2 e- -> Cu(s)} using the following potentials:
	
	\ch{Cu+ + e- -> Cu} \hspace{2em}$E^{\std} = 0.52~V$
	
	\ch{Cu^{2+} + e- -> Cu+} \hspace{2em}$E^{\std} = 0.16V$
	
	
\end{description}

\newpage
\pagestyle{empty}
\addtocounter{page}{-1}	
\section*{\emph{Pioneers}}
\paragraph{By Carol Lynn Pearson}~
\begin{verse}
	My people were Mormon pioneers.\\	
	Is the blood still good?\\	
	They stood in awe as truth\\
	Flew by like a dove\\
	And dropped a feather in the West.\\
	Where truth flies you follow\\	
	If you are a pioneer.\\
	I have searched the skies\\
	And now and then\\
	Another feather has fallen.\\
	I have packed the handcart again\\	
	Packed it with the precious things\\
	And thrown away the rest.\\
	I will sing by the fires at night\\	
	Out there on uncharted ground\\
	Where I am my own captain of tens\\	
	Where I blow the bugle\\
	Bring myself to morning prayer\\
	Map out the miles\\
	And never know when or where\\
	Or if at all I will finally say,\\
	“This is the place,”\\
	I face the plains\\
	On a good day for walking.\\
	The sun rises\\
	And the mist clears.\\
	I will be all right:\\
	My people were Mormon Pioneers.
\end{verse}

\end{document}