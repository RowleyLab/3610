\documentclass[12pt, openany, letterpaper]{memoir}
\usepackage{HomeworkStyle}

\begin{document}

\begin{center}
	{\large Homework 2 -- The First Law}
\end{center}

Name: \rule[-.1mm]{15em}{0.1pt}

\begin{description}
	\item [Exercise 2A.4(a)] ~ (5 points)

	      A sample consisting of $1.00~mol$ \ch{Ar} is expanded isothermally at $20~^\circ C$ from $10.0~dm^3$ to $30.0~dm^3$ (i) reversibly, (ii) against a constant external pressure equal to the final pressure of the gas, and (iii) freely (against zero external pressure). For the three processes calculate $q$, $w$, and $\Delta U$.

	      \vspace{8em}
	\item [Exercise 2A.5(a)] ~ (5 points)

	      A sample consisting of $1.00~mol$ of perfect gas atoms, for with $C_{V,m}=\dfrac{3}{2}R$, initially at $p_1 = 1.00~atm$ and $T_1=300~K$, is heated reversibly to $400.0~K$ at constant volume. Calculate the final pressure, $\Delta U$, $q$, and $w$.

	      \vspace{10em}
	\item [Exercise 2B.3(a)] ~ (5 points)

	      When $3.0~mol$ \ch{O2} is heated at a constant pressure of $3.25~atm$, its temperature increases from $260~K$ to $285~K$. Given that the molar heat capacity of \ch{O2} at constant pressure is $29.4\frac{J}{mol~K}$, calculate $q$, $\Delta H$, and $\Delta U$.

	      \vspace{12em}
	\item [Exercise 2C.3(b)] ~ (10 points)

	      From the following data, determine $\Delta_fH^{\std}$ for diborane, \ch{B2H6(g)}, at $298~K$:

	      \begin{tabular}{cll}
		      \circled{$1$} & \ch{B2H6(g) + 3 O2(g) -> B2O3(s) + 3 H2O(g)} & $\Delta_{rxn}H^{\std} = -1941\nicefrac{kJ}{mol}$  \\
		      \circled{$2$} & \ch{2 B(s) + 3/2 O2(g) -> B2O3(s)}           & $\Delta_{rxn}H^{\std} = -2368\nicefrac{kJ}{mol}$  \\
		      \circled{$3$} & \ch{H2(g) + 1/2 O2(g) -> H2O(g)}             & $\Delta_{rxn}H^{\std} = -241.8\nicefrac{kJ}{mol}$
	      \end{tabular}

	      \vspace{20em}
	\item [Exercise 2D.1(a)] ~ (10 points)

	      Estimate the internal pressure, $\pi_T$, of water vapor at $1.00~bar$ and $400.0~K$, treating it as a van der Waals gas, where $\pi_T=\frac{a}{V_m^2}$. \emph{Hint}: Simplify the approach by estimating the molar volume by treating the gas as perfect.

	      \vspace{20em}
	\item [Exercise 2D.4(a)] ~ (5 points)

	      The isothermal compressibility of water at $293~K$ is $4.96\times10^{-6}atm^{-1}$. Calculate the pressure that must be applied in order to increase its density by $0.10~\%$.

	      \vspace{12em}
	\item [Discussion Question 2E.1] ~ (5 points)

	      Why are adiabats steeper than isotherms?

	      \vspace{10em}
	\item [Exercise 2E.3(a)] ~ (5 points)

	      A sample consisting of $1.0~mol$ of perfect gas molecules with $C_V=20.8\frac{J}{K}$ is initially at $4.25~atm$ and $300.0~K$. It undergoes reversible adiabatic expansion until its pressure reaches $2.50~atm$. Calculate the final volume and temperature and the work done.
\end{description}

\newpage
\pagestyle{empty}
\addtocounter{page}{-1}
\newgeometry{margin=1.25in}
\section*{\emph{Holy Sonnets: Death, be not proud}}
\paragraph{By John Donne}~
\begin{verse}
	Death, be not proud, though some have called thee\\
	Mighty and dreadful, for thou art not so;\\
	For those whom thou think'st thou dost overthrow\\
	Die not, poor Death, nor yet canst thou kill me.\\
	From rest and sleep, which but thy pictures be,\\
	Much pleasure; then from thee much more must flow,\\
	And soonest our best men with thee do go,\\
	Rest of their bones, and soul's delivery.\\
	Thou art slave to fate, chance, kings, and desperate men,\\
	And dost with poison, war, and sickness dwell,\\
	And poppy or charms can make us sleep as well\\
	And better than thy stroke; why swell'st thou then?\\
	One short sleep past, we wake eternally\\
	And death shall be no more; Death, thou shalt die.
\end{verse}
\end{document}
