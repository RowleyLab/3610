\documentclass[12pt, openany, letterpaper]{memoir}
\usepackage{HomeworkStyle}
\geometry{margin=1in}

\begin{document}

\begin{center}
	{\large Homework 17 -- Chemical Kinetics}
\end{center}

Name: \rule[-.1mm]{15em}{0.1pt}

\begin{description}	
	\item [Exercise 17A.5(a)] ~ (10 points)
	
	The rate law for the reaction \ch{A + 2 B -> 3 C + D} was found to be $v=k_r[A][B]$. What are the units of $k_r$ when the concentrations are in $\nicefrac{moles}{L}$? Express the rate law in terms of (i) the rate of formation of C and (ii) the rate of consumption of A
	
	\vspace{10em}
	\item [Problem 17A.3] ~ (10 points)
	
	
	The following kinetic data ($v_0$ is the initial rate) were obtained for the reaction \\\ch{2 ICl(g) + H2(g) -> I2(g) + 2 HCl(g)}
	
	\begin{tabular}{cccc}
		Experiment & $\left[\ch{ICl}\right]_0 (\nicefrac{mmol}{L})$ & $\left[\ch{H2}\right]_0 (\nicefrac{mmol}{L})$ & $v_0 (\nicefrac{mol}{L~s})$ \\ \midrule
		$1$ & $1.5$ & $1.5$ & $3.7\times10^{-7}$\\
		$2$ & $3.0$ & $1.5$ & $7.4\times10^{-7}$\\
		$3$ & $3.0$ & $4.5$ & $22\times10^{-7}$\\
		$4$ & $4.7$ & $2.7$ & ?\\
	\end{tabular}
	
	(a) Write the rate law for the reaction. \\
	(b) From the data, determine the value of the rate constant. \\
	(c) Use the data to predict the reaction rate for experiment 4.
	
	\vspace{18em}
	\item [Exercise 17C.0(a)] ~ (5 points)
	
	In a temperature-jump experiment to investigate the kinetics of an isomerization reaction that is first order in both directions, the relaxation time was measured as $27.6\mu s$. The rate constant for the forward reaction is known to be $12.4 ms^{-1}$. Calculate the rate constant for the reverse reaction.
	
	\vspace{10em}
	\item [Exercise 17D.2(a)] ~ (10 points)
	
	The rate constant for the decomposition of a certain substance is $3.80\times10^{-3}\nicefrac{L}{mol~s}$ at $35^\circ C$ and $2.67\times10^{-2}\nicefrac{L}{mol~s}$ at $50^\circ C$. Evaluate the Arrhenius parameters of the reaction.
	
	\newpage
	\item [Discussion Question 17F.1] ~ (5 points)
	
	Discuss the conditions under which the expression (below)   for the effective rate constant of a unimolecular reaction according to the Lindemann-Hinshelwood mechanism results in a (a) first-order, or (b) second-order rate law.
	
	$k_r=\dfrac{k_ak_b\left[\ch{A}\right]}{k_b + k^\prime_a\left[\ch{A}\right]}$

	\vspace{8em}
	\item [Exercise 17F.4(a)] ~ (5 points)
	
	The enzyme-catalyzed conversion of a substrate at $25^\circ C$ has a Michaelis constant of $0.046M$. The rate of the reaction is $1.04\nicefrac{mM}{s}$ when the substrate concentration is $0.105 M$. What is the maximum velocity of this reacion?
	
	\vspace{15em}
	\item [Exercise 17G.2(a)] ~ (5 points)
	
	A substance has a fluorescence quantum yield of $\phi_{F,0}=0.35$. In an experiment to measure the fluorescence lifetime of this substance, it was observed that the fluorescence emission decayed with a half-life of $5.6ns$. What is the fluorescence rate constant of the substance?
\end{description}
\newpage
\pagestyle{empty}
\addtocounter{page}{-1}
\newgeometry{hmargin=1in, vmargin=0.85in}	
\section*{\emph{Wild Geese}}
\paragraph{By Mary Oliver}~
\begin{verse}
	You do not have to be good.\\
	You do not have to walk on your knees\\
	for a hundred miles through the desert repenting.\\
	You only have to let the soft animal of your body\\
	love what it loves.\\
	Tell me about despair, yours, and I will tell you mine.\\
	Meanwhile the world goes on.\\
	Meanwhile the sun and the clear pebbles of the rain\\
	are moving across the landscapes,\\
	over the prairies and the deep trees,\\
	the mountains and the rivers.\\
	Meanwhile the wild geese, high in the clean blue air,\\
	are heading home again.\\
	Whoever you are, no matter how lonely,\\
	the world offers itself to your imagination,\\
	calls to you like the wild geese, harsh and exciting -\\
	over and over announcing your place\\
	in the family of things.
\end{verse}

\vspace{0.75em}
\hspace{12em}
\begin{minipage}{0.8\linewidth}
\section*{\emph{The Summer Day}}
\paragraph{By Mary Oliver}~
\begin{verse}
	Who made the world?\\
	Who made the swan, and the black bear?\\
	Who made the grasshopper?\\
	This grasshopper, I mean-\\
	the one who has flung herself out of the grass,\\
	the one who is eating sugar out of my hand,\\
	who is moving her jaws back and forth instead of up and down-\\
	who is gazing around with her enormous and complicated eyes.\\
	Now she lifts her pale forearms and thoroughly washes her face.\\
	Now she snaps her wings open, and floats away.\\
	I don't know exactly what a prayer is.\\
	I do know how to pay attention, how to fall down\\
	into the grass, how to kneel down in the grass,\\
	how to be idle and blessed, how to stroll through the fields,\\
	which is what I have been doing all day.\\
	Tell me, what else should I have done?\\
	Doesn't everything die at last, and too soon?\\
	Tell me, what is it you plan to do\\
	with your one wild and precious life?
\end{verse}
\end{minipage}
\end{document}