\documentclass[12pt, openany, letterpaper]{memoir}
\usepackage{HomeworkStyle}

\begin{document}

\begin{center}
	{\large Homework 3 -- The Second and Third Laws}
	
	Due: Friday, September 29 \hspace{3em} Points: ${\dfrac{~}{~~50~~}}$
\end{center}

Name: \rule[-.1mm]{15em}{0.1pt}

\begin{description}	
	\item [Exercise 3A.2(a)] ~ (5 points)
	
	A certain ideal heat engine uses water that ht triple point as the hot source and an organic liquid as the cold sink. It withdraws $10.00~kJ$ of heat from the hot source and generates $3.00~kJ$ of work. What is the temperature of the organic liquid?
	
	\vspace{15em}
	\item [Exercise 3A.5(a)] ~ (5 points)
	
	Calculate the change in entropy when $15~g$ of carbon dioxide gas is allowed to expand from $1.0~dm^3$ to $3.0~dm^3$ at $300~K$.
	
	\vspace{15em}\newpage
	\item [Exercise 3A.8(b)] ~ (10 points)
	
	Calculate $\Delta S$ (for the system) when the state of $2.00~mol$ of diatomic perfect gas molecules, for which $C_{p,m}=\dfrac{5}{2}R$, is changed from $25^\circ C$ and $1.50~atm$ to $135^\circ C$ and $7.00~atm$. How do you rationalize the sign of $\Delta S$?
	
	\vspace{21em}

	\item [Exercise 3A.9(a)] ~ (10 points)
	
	Calculate $\Delta S_{tot}$ when two copper blocks, each of mass $1.00~kg$, one at $50^\circ C$ and the other at $0^\circ C$ are placed in contact in an isolated container. The specific heat capacity of copper is $0.385~\nicefrac{J}{gK}$ and may be assumed constant over the temperature range involved.
	
	\vspace{21em}
	
	\item [Exercise 3C.3(a)] ~ (10 points)
	
	Calculate the maximum non-expansion work per mole that may be obtained from a fuel cell in which the chemical reaction is the combustion of methane at $298~K$

	\vspace{20em}
	
	\item [Exercise 3D.1(a)] ~ (5 points)
	
	Suppose that $2.5~mmol~\ch{N2(g)}$ occupies $42~cm^3$ at $300~K$ and expands isothermally to $600~cm^3$. Calculate $\Delta G$ for the process
	
	\vspace{12em}
	\item [Exercise 3D.4(a)] ~ (5 points)
	
	Calculate the change in the molar Gibbs energy of hydrogen gas when its pressure is increased isothermally from $1.0~atm$ to $100.0~atm$ at $298~K$.
	

\end{description}
\end{document}