\documentclass[12pt, openany, letterpaper]{memoir}
\usepackage{HomeworkStyle}

\begin{document}

\begin{center}
	{\large Homework 3 -- The Second and Third Laws}
\end{center}

Name: \rule[-.1mm]{15em}{0.1pt}

\begin{description}	
	\item [Exercise 3A.2(a)] ~ (5 points)
	
	A certain ideal heat engine uses water at the triple point as the hot source and an organic liquid as the cold sink. It withdraws $10.00~kJ$ of heat from the hot source and generates $3.00~kJ$ of work. What is the temperature of the organic liquid?
	
	\vspace{15em}
	\item [Exercise 3A.5(a)] ~ (5 points)
	
	Calculate the change in entropy when $15~g$ of carbon dioxide gas is allowed to expand from $1.0~dm^3$ to $3.0~dm^3$ at $300~K$.
	
	\vspace{15em}\newpage
	\item [Exercise 3A.8(b)] ~ (10 points)
	
	Calculate $\Delta S$ (for the system) when the state of $2.00~mol$ of diatomic perfect gas molecules, for which $C_{p,m}=\dfrac{7}{2}R$, is changed from $25^\circ C$ and $1.50~atm$ to $135^\circ C$ and $7.00~atm$. How do you rationalize the sign of $\Delta S$?
	
	\vspace{21em}

	\item [Exercise 3A.9(a)] ~ (10 points)
	
	Calculate $\Delta S_{tot}$ when two copper blocks, each of mass $1.00~kg$, one at $50^\circ C$ and the other at $0^\circ C$ are placed in contact in an isolated container. The specific heat capacity of copper is $0.385~\nicefrac{J}{gK}$ and may be assumed constant over the temperature range involved.
	
	\vspace{21em}
	
	\item [Exercise 3C.3(a)] ~ (10 points)
	
	Calculate the maximum non-expansion work per mole that may be obtained from a fuel cell in which the chemical reaction is the combustion of methane at $298~K$

	\vspace{20em}
	
	\item [Exercise 3D.1(a)] ~ (5 points)
	
	Suppose that $2.5~mmol~\ch{N2(g)}$ occupies $42~cm^3$ at $300~K$ and expands isothermally to $600~cm^3$. Calculate $\Delta G$ for the process
	
	\vspace{12em}
	\item [Exercise 3D.4(a)] ~ (5 points)
	
	Calculate the change in the molar Gibbs energy of hydrogen gas when its pressure is increased isothermally from $1.0~atm$ to $100.0~atm$ at $298~K$.
	

\end{description}

\newpage
\pagestyle{empty}
\addtocounter{page}{-1}
\section*{\emph{Still I Rise}}
\paragraph{By Maya Angelou}~

\vspace{1em}
\begin{minipage}[t]{0.49\linewidth}
	
	You may write me down in history\\
	With your bitter, twisted lies,\\
	You may trod me in the very dirt\\
	But still, like dust, I'll rise.
	
	Does my sassiness upset you?\\
	Why are you beset with gloom?\\
	’Cause I walk like I've got oil wells\\
	Pumping in my living room.
	
	Just like moons and like suns,\\
	With the certainty of tides,\\
	Just like hopes springing high,\\
	Still I'll rise.
	
	Did you want to see me broken?\\
	Bowed head and lowered eyes?\\
	Shoulders falling down like teardrops,\\
	Weakened by my soulful cries?
	
	Does my haughtiness offend you?\\
	Don't you take it awful hard\\
	’Cause I laugh like I've got gold mines\\
	Diggin’ in my own backyard.
	
	You may shoot me with your words,\\
	You may cut me with your eyes,\\
	You may kill me with your hatefulness,\\
	But still, like air, I’ll rise.	
	
\end{minipage}
\begin{minipage}[t]{0.49\linewidth}
	
	Does my sexiness upset you?\\
	Does it come as a surprise\\
	That I dance like I've got diamonds\\
	At the meeting of my thighs?
	
	Out of the huts of history’s shame\\
	I rise\\
	Up from a past that’s rooted in pain\\
	I rise\\
	I'm a black ocean, leaping and wide,\\
	Welling and swelling I bear in the tide.
	
	Leaving behind nights of terror and fear\\
	I rise\\
	Into a daybreak that’s wondrously clear\\
	I rise\\
	Bringing the gifts that my ancestors gave,\\
	I am the dream and the hope of the slave.\\
	I rise\\
	I rise\\
	I rise.	
	
\end{minipage}
\end{document}