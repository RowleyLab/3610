\documentclass[12pt, openany, letterpaper]{memoir}
\usepackage{HomeworkStyle}

\begin{document}

\begin{center}
	{\large Homework 19 -- Molecules in Motion}
	
	Due: Friday, November 10 \hspace{3em} Points: ${\dfrac{~}{~~45~~}}$
\end{center}

Name: \rule[-.1mm]{15em}{0.1pt}

\begin{description}	
	\item [Exercise 19A.2(a)] ~ (10 points)
	
	Calculate the diffusion constant of argon at $20~^\circ C$ and (i) $1.00~Pa$, (ii) $100.0~kPa$, and (iii) $10.0~MPa$. If a pressure gradient of $1.0~\nicefrac{bar}{m}$ is established in a pipe, what is the flow of gas due to diffusion?
	
	
	\vspace{19em}
	\item [Exercise 19A.6(a)] ~ (5 points)
	
	Use the experimental value of the coefficient of viscosity for neon (Table 19A.1) to estimate the collision cross-section of \ch{Ne} atoms at $273~K$.
	
	\vspace{20em}
	\item [Exercise 19B.1(a)] ~ (5 points)
	The viscosity of water at $20~^\circ C$ is $1.002~cP$ and $0.7975~cP$ at $30~^\circ C$. What is the energy of activation for the transport process?
	
	\vspace{23em}
	\item [Exercise 19B.3(a)] ~ (5 points)
	
	The mobility of a \ch{Rb^{+}} ion in aqueous solution is $7.92\times10^{-8}\nicefrac{m^2}{Vs}$ at $25~^\circ C$. The potential difference between two electrodes placed in the solution is $25.0~V$. If the electrodes are $7.00~mm$ apart, what is the drift speed of the \ch{Rb^{2+}} ion?
	
	\vspace{23em}	
	\item [Exercise 19B.6(a)] ~ (5 points)
	
	Estimate the effective radius of a sucrose molecule in water at $25~^\circ C$ given that its diffusion coefficient is $5.2\times10^{-10}\nicefrac{m^2}{s}$ and the viscosity of water is $1.00~cP$
	
	\vspace{23em}	
	\item [Exercise 19C.5(a)] ~ (5 points)	
	
	The diffusion coefficient of \ch{CCl4} in heptane at $25~^\circ C$ is $3.17\times10^{-9}~\nicefrac{m^2}{s}$. Estimate the time required for a \ch{CCl4} molecule to have a root-mean-square displacement of $5.0~mm$.	
	
	\vspace{23em}	
	\item [Exercise 19C.2(a)] ~ (10 points)	

	A layer of $20.0~g$ of sucrose is spread uniformly over a surface of area $5.0~cm^2$ and covered in water to a depth of $20~cm$. What will be the molar concentration of sucrose molecules at $10~cm$ above the original layer at (i) $10~s$, and (ii) $24~h$? Assume diffusion is the only transport process and take $D=5.216\times10^{-9}~\nicefrac{m^2}{s}$.
	


\end{description}
\end{document}