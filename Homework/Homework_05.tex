\documentclass[12pt, openany, letterpaper]{memoir}
\usepackage{HomeworkStyle}

\begin{document}

\begin{center}
	{\large Homework 5 -- Simple Mixtures}
	
\end{center}

Name: \rule[-.1mm]{15em}{0.1pt}

\begin{description}	
	\item [Exercise 5A.8(a)] ~ (5 points)
	
	At $300~K$, the partial vapor pressure of \ch{HCl} (that is, the partial pressure of the \ch{HCl} vapor) in liquid \ch{GeCl4} is as follows:
	
	\begin{tabular}{c|c|c|c}
		$\chi_{\ch{HCl}}$	& 	$0.005$	&	$0.012$	&	$0.019$\\
		$p_{\ch{HCl}}(kPa)$	&	$32.0$	&	$76.9$	&	$121.8$
	\end{tabular}

	Show that the solution obeys Henry's law in this range of mole fractions, and calculate the Henry's law constant at $300~K$
	
	\vspace{15em}
	\item [Exercise 5B.2(a)] ~ (5 points)
	
	The vapor pressure of benzene is $53.3~kPa$ at $60.6^\circ C$, but it fell to $51.5~kPa$ when $19.0~g$ of a non-volatile organic copound was dissolved in $500~g$ of benzene. Calculate the molar mass of the compound

	
	\vspace{25em}
	\item [Exercise 5B.8(a)] ~ (5 points)
	
	The enthalpy of fusion of anthracene is $28.8\frac{kJ}{mol}$ and its melting point is $217^\circ C$. Calculate its ideal solubility in benzene at $25^\circ C$.
	
	\vspace{20em}
	\item [Exercise 5C.3(a)] ~ (5 points)
	
	Phenol and water form non-ideal liquid mixtures. When $7.32~g$ of phenol and $7.95~g$ of water are mixed together at $60^\circ C$ they form two immiscible liquid phases with mole fractions of phenol of $0.042$  and $0.161$. (i) Calculate the overall mole fraction of the phenol in the mixture. (ii) Use the lever rule to determine the relative amounts of the two phases.
	
	\vspace{25em}
	\item [Exercise 5F.2(a)] ~ (5 points)
	
Substances $A$ and $B$ are both volativel liquids with $p_A^\star=300~Torr$, $p_B^\star=250~Torr$, and $K_B=200~Torr$ (For concentration expressed in mole fraction). When $\chi_A=0.900$, $p_A=250~Torr$, and $p_B=25~Torr$. Calculate the activities of $A$ and $B$. Use the mole fraction, Raoult's law basis system for $A$ and the Henry's law basis system for $B$. Go on to calculate the activity coefficients of $A$ and $B$
	
	\vspace{20em}
	\item [Exercise 5F.7(a)] ~ (5 points)
	
	Estimate the mean ionic activity coefficient ($\gamma_\pm$) and activity of \ch{CaCl2} in a solution that is $0.010~\nicefrac{mol}{kg}$ \ch{CaCl2(aq)} and $0.030~\nicefrac{mol}{kg}$ \ch{NaF(aq)} at $25^\circ C$.
	
	\vspace{18em}

	
\newpage
\pagestyle{empty}
\addtocounter{page}{-1}	
\section*{\emph{Slaveships}}
\paragraph{By Lucille Clifton}~
\begin{verse}
	loaded like spoons\\
	into the belly of Jesus\\
	where we lay for weeks for months\\
	in the sweat and stink\\
	of our own breathing\\
	Jesus\\
	why do you not protect us\\
	chained to the heart of the Angel\\
	where the prayers we never tell\\
	and hot and red\\
	as our bloody ankles\\
	Jesus\\
	Angel\\
	can these be men\\
	who vomit us out from ships\\
	called Jesus    Angel    Grace of God\\
	onto a heathen country\\
	Jesus\\
	Angel\\
	ever again\\
	can this tongue speak\\
	can these bones walk\\
	Grace Of God\\
	can this sin live
\end{verse}
\end{description}
\end{document}