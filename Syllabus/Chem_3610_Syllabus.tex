\documentclass[12pt, letterpaper]{article}
\usepackage{SyllabusStyle}

\begin{document}
\begin{center}
	{\Large \textsc{Physical Chemistry I}}

	CHEM 3610
\end{center}
\begin{center}
	{\large Fall 2022}
\end{center}
\begin{center}
	\rule{0.99\textwidth}{0.4pt}
	\begin{tabular}{llcll}
		\textbf{Instructor:} & Matthew Rowley           &  & \textbf{Office Hours:} & MWF 10:00 am -- 11:00 am \\
		\textbf{Telephone:}  & (435) 586-7875           &  &                        & TR 1:00 pm -- 2:00 pm    \\
		\textbf{Email:}      & matthewrowley$1$@suu.edu &  & \textbf{Office:}       & SC-220                   \\
		\multicolumn{5}{c}{Please include the course number in the subject line of all correspondence.}
	\end{tabular}
	\rule{0.99\textwidth}{0.4pt}
\end{center}

\section*{Course Description}
This is the first of two courses in physical chemistry, together with Chem 3620. In this class we will study the behavior of large ensembles rather than the characteristics of individual particles. These behaviors are emergent based on the energetic details of the system. Topics are categorized broadly as thermodynamics, kinetics, and statistical mechanics, and range in scope from fundamental theories to practical application in fields like engineering and medicine.

\paragraph{Prerequisites:}~

A minimum grade of C (2.0 or above) in CHEM 1220/1225 and MATH 1220

Admission to the Chemistry program

\paragraph{Concurrent requisite:}
CHEM 3615 -- Physical Chemistry Lab I

\paragraph{Course Materials:} ~

$\circ$ \emph{Physical Chemistry: Thermodynamics, Structure, and Change} by Atkins, de Paula, and Keeler -- 11th edition (required) ISBN: 978-0-19-876986-6

$\circ$ \emph{Applied Mathematics for Physical Chemistry} by James R. Barrante (suggested)

\vspace{-4pt}
---or---

\vspace{-4pt}
$\circ$ \emph{Mathematics for Physical Chemistry} by Robert G. Mortimer (suggested)


\section*{Tentative Schedule}
This class will meet on Mondays, Wednesdays, and Fridays from 12:00 to 12:50 pm in room 127 of the Science Center (SC).

\noindent For the best lecture experience, read the indicated textbook chapter \emph{before} each lecture.

\noindent
\begin{tabular}{rcccc}
& Date && Topic & Chapter\\
\midrule
Week 1 & W, Aug. 31&& The Perfect Gas and Kinetic Model & 1A-1B\\
& F, Sep. 2&& Real Gases & 1C\\
\midrule
Week 2 & M, Sep. 5& \multicolumn{3}{l}{\textbf{Labor Day - No Class!}}\\
& W, Sep. 7&& Internal Energy & 2A\\
& F, Sep. 9&& Enthalpy & 2B\\
\midrule
Week 3 & M, Sep. 12&& Thermochemistry & 2C\\
& W, Sep. 14&& State Functions and Exact Differentials & 2D\\
& F, Sep. 16&& Adiabatic Changes & 2E\\
\midrule
Week 4 & M, Sep. 19& \multicolumn{3}{l}{\textbf{Makeup/Review Day (Midterm 1: Ch. 1--2)}}\\
& W, Sep. 21&& Entropy & 3A\\
& F, Sep. 23&& Entropy Changes & 3B\\
\midrule
Week 5 & M, Sep. 26&& The Measurement of Entropy and the System & 3C-3D\\
& W, Sep. 28&& Combining the First and Second Laws & 3D\\
& F, Sep. 30&& Phase Diagrams of Pure Substances & 4A\\
\midrule
Week 6 & M, Oct. 3&& Thermodynamic Aspects of Phase Transitions & 4B\\
& W, Oct. 5& \multicolumn{3}{l}{\textbf{Makeup/Review Day (Midterm 2: Ch. 3--4)}}\\
& F, Oct. 7&& The Thermodynamic Description of Mixtures & 5A\\
\midrule
Week 7 & M, Oct. 10&& The Properties of Solutions & 5B\\
& W, Oct. 12&& Phase Diagrams of Binary Systems & 5C-5D\\
& F, Oct. 14&& Phase Diagrams of Ternary Systems & 5E\\
\midrule
Week 8 & M, Oct. 17& \multicolumn{3}{l}{\textbf{Fall Break - No Class!}}\\
& W, Oct. 19&& Activities & 5F\\
& F, Oct. 21&& The Equilibrium Constant & 6A\\
\end{tabular}

\noindent
\begin{tabular}{rcccc}
& Date && Topic & Chapter\\
\midrule
Week 9 & M, Oct. 24&& The Response of Equilibria to the Conditions & 6B\\
& W, Oct. 26&& Electrochemical Cells & 6C-6D\\
& F, Oct. 28&& Transport in Gases & 16A\\
\midrule
Week 10 & M, Oct. 31&& Motion in Liquids & 16B\\
& W, Nov. 2&& Diffusion & 16C\\
& F, Nov. 4& \multicolumn{3}{l}{\textbf{Makeup/Review Day (Midterm 3: Ch. 5, 6, 16)}}\\
\midrule
Week 11 & M, Nov. 7&& The Rates of Chemical Reactions & 17A\\
& W, Nov. 9&& Integrated Rate Laws & 17B\\
& F, Nov. 11&& Reactions Approaching Equilibrium & 17C\\
\midrule
Week 12 & M, Nov. 14&& The Arrhenius Equation & 17D\\
& W, Nov. 16&& Reaction Mechanisms & 17E-17F\\
& F, Nov. 18&& Photochemistry & 17G\\
\midrule
Week 13 & M, Nov. 21& \multicolumn{3}{l}{\textbf{Thanksgiving Break - No Class!}}\\
& W, Nov. 23& \multicolumn{3}{l}{\textbf{Thanksgiving Break - No Class!}}\\
& F, Nov. 25& \multicolumn{3}{l}{\textbf{Thanksgiving Break - No Class!}}\\
\midrule
Week 14 & M, Nov. 28&& Collision Theory & 18A\\
& W, Nov. 30&& Diffusion-Controlled Reactions & 18B\\
& F, Dec. 2&& Transition-State Theory & 18C\\
\midrule
Week 15 & M, Dec. 5&& The Dynamics of Molecular Collisions & 18D\\
& W, Dec. 7&& Electron Transfer in Homogeneous Systems & 18E\\
& F, Dec. 9& \multicolumn{3}{l}{\textbf{Makeup/Review Day (Midterm 4: Ch. 17--18)}}\\
	\midrule
	\midrule
	Finals Week & W, Dec. 14 & \multicolumn{3}{l}{\textbf{Final Exam} 11:00--12:50  Bring a pencil and a scantron!}                                                      \\
\end{tabular}


\section*{Course Policies and Grading}
Grades will be based on the following items:
\begin{description}
	\item[4 Midterm Exams] 40\%
	\item[Final Exam] 15\%
	\item[Quizzes] 15\%
	\item[Homework] 30\%
\end{description}
Final Grades will be assigned according to the following grade scale:

\begin{tabular}{cl|c|cl}
	Percentage & Grade &  & Percentage & Grade \\ \midrule
	100--93.0  & A     &  & 77.0--73.0 & C     \\
	93.0--90.0 & A-    &  & 73.0--70.0 & C-    \\
	90.0--87.0 & B+    &  & 70.0--67.0 & D+    \\
	87.0--83.0 & B     &  & 67.0--63.0 & D     \\
	83.0--80.0 & B-    &  & 63.0--60.0 & D-    \\
	80.0--77.0 & C+    &  & < 60.0     & F
\end{tabular}
\paragraph{Midterm Exams:}
There are four midterm exams, to be completed in class on the designated day unless prior arrangements have been made. It is departmental policy that exams not be returned, although students may examine their completed exam and the answer key in my office. Scantron sheets are not required for midterms.

\paragraph{Final Exam:}
The final exam is comprehensive. The final is produced by the American Chemical Society, and the instructor will not have access to the exam prior to its administration. Therefore, it is to your advantage to learn as much as possible throughout the semester. The test is multiple choice and a scantron will be required.

\paragraph{Homework:}
The homework assignments will be a combination of problems from the textbook as well as problems that I have written or gathered from various other resources. These assignments are designed to be somewhat challenging and will require diligence in order to complete. They sometimes push the students to think \emph{beyond} the lecture material.

\paragraph{Quizzes:}
Quizzes will be shorter and more frequent assignments than the regular homework. The difficulty and scope of the questions on quizzes will more closely match what you will see on the exams.

\paragraph{Attendance Policy:}
Students are expected to attend class. If you must miss class, contact the instructor.

\paragraph{Late Work Policy:}
Homework and take-home quizzes will be due on a day when class is regularly scheduled. All work is to be turned in at the \emph{beginning} of the class period, and late work will not be accepted.

\paragraph{Make-up Work Policy:}
In general, there will be no opportunity to make up missed work, including in-class quizzes. If you must miss class, please do any assigned work in advance, and arrange to turn it in early.

\section*{Miscellany}
\paragraph{Important syllabus statements related to ATTENDANCE and COVID-19:} ~

\noindent\emph{What should I expect in the classroom this semester?}

\noindent The following are general guidelines for the classroom environment
\begin{description}
	\item[Class Attendance is Required:] If you are registered for a Face-to-Face, Synchronous Remote, or Hybrid course, attendance is required. If you are ill or instructed to isolate or quarantine, you may request a faculty member record the class and share it with you or you may request other reasonable accommodations. Your instructor will work with you to develop a plan for completing coursework while you are isolated/quarantined. In order for you to receive academic accommodations and ensure that your request is communicated to faculty, you must submit this \href{https://my.suu.edu/covid/selfreport/}{self report form}.
	\item[\href{https://www.suu.edu/registrar/onlinehybrid.html}{Course ~delivery ~modalities} ~are ~posted ~online ~for ~each ~course, ~but ~may ~be ~modified ~in] \textbf{response to emerging COVID conditions:} SUU is employing every effort to maintain a learning environment that is engaging and safe. The course modality listed when you registered for courses should remain for the semester; however, due to COVID conditions, the delivery of modality for a specific course may change during the semester. Normally, these changes will be short term (possibly the length of a quarantine or isolation time period), or in some cases longer. When such a modification is needed, faculty members will work with their department chair and/or dean and the students to maintain an effective learning environment.
\end{description}

\paragraph{Scientific Calculator:}
There are many different ways to calculate figures during homework. It is tempting to rely on Online resources such as \href{http://www.wolframalpha.com}{http://www.wolframalpha.com}, or to simply use a calculator application on a smart phone. During exams, however, any devices capable of connecting to the Internet will \emph{not} be allowed. You will instead need a scientific calculator capable of performing exponentiation and logarithms for the exams. Using this calculator exclusively while doing homework will ensure that you are familiar with it for use during exams.

\paragraph{Academic Integrity:}
Scholastic dishonesty will not be tolerated and will be prosecuted to the fullest extent (see \href{https://www.suu.edu/policies/06/33.html}{SUU Policy 6.33}). You are expected to have read and understood the current SUU student conduct code (\href{https://www.suu.edu/policies/11/02.html}{SUU Policy 11.2}) regarding student responsibilities and rights, the intellectual property policy (\href{https://www.suu.edu/policies/05/52.html}{SUU Policy 5.52}), information about procedures, and what constitutes acceptable behavior.

\paragraph{Mental Health:}
Mental and physical health are equal components to a holistic view of wellness and human thriving. Mental health should not be ignored, dismissed, or demeaned. If you find yourself struggling with mental health please visit \href{https://www.suu.edu/mentalhealth}{https://www.suu.edu/mentalhealth} for resources. There is also a link prominently on the right side of every Canvas page.

\paragraph{ADA Policy:}
Students with medical, psychological, learning, or other disabilities desiring academic adjustments, accommodations, or auxiliary aids will need to contact the Southern Utah University Coordinator of Services for Students with Disabilities (SSD), in Room 206F of the Sharwan Smith Center or phone (435) 865-8022. SSD determines eligibility for and authorizes the provision of services.

\paragraph{Emergency Management Statement:}
In case of emergency, the university's Emergency Notification System (ENS) will be activated. Students are encouraged to maintain updated contact information using the link on the homepage of the \emph{mySUU} portal. In addition, students are encouraged to familiarize themselves with the Emergency Response Protocols posted in each classroom. Detailed information about the university's emergency management plan can be found at: \href{http://www.suu.edu/emergency}{http://www.suu.edu/emergency}

\paragraph{HEOA Compliance Statement:}
The sharing of copyrighted material through peer-to- peer (P2P) file sharing, except as provided under U.S. copyright law, is prohibited by law. Detailed information can be found at: \href{https://help.suu.edu/article/1097/p2p-and-copyright-infringement}{https://help.suu.edu/article/1097/p2p-and-copyright-infringement}

\paragraph{LINK Statement:}
SUU faculty and staff care about the success of our students. In addition to your professor, numerous services are available to assist you with the achievement of your educational goals. SUU's LINK system may be used by faculty to notify you and/or your advisors of their concern for your progress and provide references to campus services as appropriate.

\paragraph{SUUSA Statement:}
As a student at SUU, you have representation from the SUU Student Association (SUUSA) which advocates for student interests and helps work as a liaison between the students and the university administration. You can submit My SUU Voice feedback by going here: \href{https://www.suu.edu/suusa/voice}{https://www.suu.edu/suusa/voice} Likewise, you can learn more about SUUSA's Executive Council here (\href{https://www.suu.edu/suusa/executive-council/}{https://www.suu.edu/suusa/executive-council/}) and about individual SUUSA's Student Senators here (\href{https://www.suu.edu/suusa/senate/}{https://www.suu.edu/suusa/senate/})

\paragraph{Land Acknowledgement Statement:}
SUU wishes to acknowledge and honor the Indigenous communities of this region as original possessors, stewards, and inhabitants of this Too’veep (land), and recognize that the University is situated on the traditional homelands of the Nung’wu (Southern Paiute People). We recognize that these lands have deeply rooted spiritual, cultural, and historical significance to the Southern Paiutes. We offer gratitude for the land itself, for the collaborative and resilient nature of the Southern Paiute people, and for the continuous opportunity to study, learn, work, and build community on their homelands here today. Consistent with the University's ongoing commitment to equity, diversity, and inclusion, SUU works towards building meaningful relationships with Native Nations and Indigenous communities through academic pursuits, partnerships, historical recognitions, community service, and student success efforts.

\paragraph{Disclaimer:}
Information contained in this syllabus, other than the grading, late assignments, make up work and attendance policies, may be subject to change as deemed appropriate by the instructor.
\end{document}
