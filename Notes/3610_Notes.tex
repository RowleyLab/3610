\documentclass[12pt, openany, letterpaper]{memoir}
\usepackage{NotesStyle}
\renewcommand\thesection{\thechapter\Alph{section}}
\renewcommand\thesubsection{\thesection.\Numeral{subsection}}


\begin{document}
\title{CHEM 3610 Lecture Notes}
\author{Matthew Rowley}
\date{Fall 2020}
\mainmatter
\maketitle

\chapter*{Course Administrative Details}
\begin{itemize}
	\item My office hours
	\item Intro to my research
	\item Introductory Quiz
	\item Grading details
	      \begin{itemize}
		      \item Exams - 40, Final - 15, Quizzes - 15, Homework - 30
		      \item Homework
		      \item Quizzes
	      \end{itemize}
	\item Importance of reading and learning on your own
	\item Learning resources
	      \begin{itemize}
		      \item My Office Hours
		      \item Tutoring services - https://www.suu.edu/academicsuccess/tutoring/
	      \end{itemize}
	\item Show how to access Canvas
	      \begin{itemize}
		      \item Calendar, Grades, Modules, etc.
		      \item Zoom, etc.
	      \end{itemize}
\end{itemize}

\chapter*{Textbook Errata}
\section*{Chapter 1}
\begin{itemize}
	\item p. 42 -- In Brief illustration 1B.2 the $v_{rms}$ is used where $v_{mean}$ should be
	\item p. 46 -- $Z = \dfrac{RT}{pV_m^\circ}$ in the paragraph between equations 1C.1 and 1C.2 is wrong. The expression for the molar volume of an ideal gas was erroneously substituted into the real gas molar volume
\end{itemize}
\section*{Chapter 2}
\begin{itemize}
	\item p.86 -- Equations 2C.6 and 2C.7a both should have ``\ldots$=H(T_1)$\ldots''  instead of ``\ldots$=H(T_2)$\ldots''
\end{itemize}
\section*{Chapter 5}
\begin{itemize}
	\item p.203 -- Equation 5C.4 should be: $y_A = \dfrac{\chi_Ap^\star_A}{p^\star_B+\left(p^\star_A-p^\star_B\right)\chi_A} = 1-y_B$
\end{itemize}

\chapter{The Properties of Gases}
\section{The Perfect Gas}
\paragraph{Variables of State}
\begin{itemize}
	\item Gases are composed of particles moving freely through space
	\item Gas particles interact only weakly, and only near the point of collision
	\item Collisions can explain all of the observed gas properties (such as temperature and pressure)
	\item Variables of state:
	      \begin{itemize}
		      \item Pressure ($p$)
		            \begin{itemize}
			            \item Pressure is the force a gas exerts on its surroundings (chamber walls)
			            \item Pressure can be measured by a barometer, manometer, or other gauge
			            \item Volume will change unless the pressure is in mechanical equilibrium with another force
		            \end{itemize}
		      \item Temperature ($T$)
		            \begin{itemize}
			            \item Temperature is a measure of the kinetic energy of the gas particles
			            \item A perfect gas can be used to determine the \emph{thermodynamic temperature scale}
			            \item This scale (called ``Kelvin'') relates to Celsius by: $T_K = T_{^\circ C} - 273.15$ (exact)
		            \end{itemize}
		      \item Volume ($V$)
		      \item Moles ($n$) -- The number of moles of gas particles in a particular gas sample
	      \end{itemize}
\end{itemize}
\paragraph{Equations of State}
\begin{itemize}
	\item An equation of state relates all of the state variables for a gas
	\item Developing the perfect gas equation of state:
	      \begin{itemize}
		      \item By keeping one variable (plus $n$) fixed, the relationship between the other two can be determined (isotherms, isochores, and isobars)
		      \item Boyle's Law: $p\propto \dfrac{1}{V}$ at constant $n,T$
		      \item Charles's Law: $V\propto T$ at constant $n,p$
		      \item Avogardro's Principle: $V\propto n$ at constant $T,p$
	      \end{itemize}
	\item The same proportionality constant consistently appears: The gas constant $R$
	\item The gas constant can also be related to the Boltzmann constant: $R=N_A k_B$
	\item Combining these relations gives the perfect gas law: $pV=nRT$
	\item For mixtures of perfect gases, each gas exerts a partial pressure as if the others were not present
	\item This is Dalton's law of partial pressures: ${p_{Total} = \sum{\chi_ip_i}}$

\end{itemize}

\section{The Kinetic Model}
\paragraph{Pressure and Molecular Speeds}
\begin{itemize}
	\item The kinetic-molecular theory of gases explains gas properties solely through elastic collisions of gas particles
	\item Despite being remarkably simple, kinetic theory accurately describes almost all physical properties and processes a gas can exhibit.
	\item This theory has three main assumptions:
	      \begin{itemize}
		      \item A gas is composed of molecules in random ceaseless motion
		      \item The size of the molecules is negligible
		      \item All molecular collisions are elastic (translational kinetic energy is conserved)
	      \end{itemize}
	\item Pressure can be determined by analyzing the momentum transferred to the walls of a container through molecular collisions
	\item This approach also leads to the perfect gas law in a different form: $pV=\dfrac{1}{3}nMv_{rms}^2$
	\item Here, $M$ is the molar molecular mass and $v_{rms}=\avg{v^2}^{\nicefrac{1}{2}}$
	\item Comparing this to the perfect gas law gives the relation: $v_{rms}=\left(\dfrac{3RT}{M}\right)^{\nicefrac{1}{2}}$
\end{itemize}
\paragraph{The Maxwell-Boltzmann Distribution of Speeds}
\begin{itemize}
	\item In a ensemble of molecules, not all of them will have the average velocity
	\item Determining the distribution of speeds was a tour-de-force for Boltzmann and Maxwell, and one of the first applications of statistical mechanics
	\item For a perfect gas: $f(v) = 4\pi\left(\dfrac{M}{2\pi RT}\right)^{\nicefrac{3}{2}}v^2e^{\frac{-Mv^2}{2RT}}$
	\item Figure 1B.4 compares the velocity distributions of molecules of different sizes and at different temperatures
\end{itemize}
\paragraph{Mean Values}
\begin{itemize}
	\item Mean Speed: $v_{mean} = \left(\dfrac{8RT}{\pi M}\right)^{\nicefrac{1}{2}} = \left(\dfrac{8}{3\pi}\right)^{\nicefrac{1}{2}}v_{rms}$
	\item Most Probable Speed: $v_{mp} = \left(\dfrac{2RT}{ M}\right)^{\nicefrac{1}{2}} = \left(\dfrac{2}{3}\right)^{\nicefrac{1}{2}}v_{rms}$
	\item Mean Relative Speed (Similar Particles): $v_{rel}=\sqrt{2}v_{mean}$
	\item Mean Relative Speed (Different Particles): $v_{rel} = \left(\dfrac{8RT}{\pi\mu}\right)^{\nicefrac{1}{2}} \hspace{2em} \mu = \dfrac{M_AM_B}{M_A+M_B}$
\end{itemize}
\paragraph{Collisions}
\begin{itemize}
	\item Figure 1B.8 provides a framework for evaluating collisions
	\item $\sigma$, the collision cross-section, for some common gases is given in Table 1B.2 (For \ch{N2}, $\sigma = 0.43~nm^2$)
	\item This gives the collision frequency as: $z=\sigma v_{rel}\mathcal{N}\hspace{2em}\mathcal{N} = \dfrac{N}{V} = \dfrac{pN_A}{RT} = \dfrac{p}{k_BT}$
	\item The mean free path ($\lambda$) is the average distance a particle travels between collisions
	\item $\lambda = \dfrac{v_{rel}}{z} = \dfrac{k_BT}{\sigma p}$ \hspace{2em} Note that for unit cancellation, $p$ should be in $Pa$
\end{itemize}

\section{Real Gases}
\paragraph{Deviations from Perfect Behavior}
\begin{itemize}
	\item Real gases interact with each other through both attractions and repulsions
	\item Under different circumstances, either of the two effects can dominate
	\item Attractions give a lower pressure than a perfect gas, while repulsions give a higher pressure than a perfect gas
	\item Figure 1C.1 shows the potential curve which is relevant as particles approach and collide
	\item At low pressures, the attractive portion of the potential dominates and the gases are more easily compressed
	\item At high pressures, the repulsive portion of the potential dominates and gases are more difficult to compress
	\item The potential curve will come up again soon in an unusual way, with respect to the Joule-Thompson coefficient
	\item Examine the path from A $\rightarrow$ F in Figure 1C.2
	      \begin{itemize}
		      \item For A $\rightarrow$ B, the gas behaves like a perfect gas
		      \item For B $\rightarrow$ C, the gas deviates from a perfect gas, requiring more pressure to compress the gas
		      \item For C $\rightarrow$ E, the gas compresses with no additional pressure increase. During this period, the gas is actually condensing out as a liquid. The gas phase remains the same, but through the compression it comprises a smaller and smaller fraction of the whole, with the liquid phase growing
		      \item For E $\rightarrow$ F, only liquid phase remains, and liquids are very incompressible
	      \end{itemize}
	\item Figure 1C.2 contains a few more interesting points:
	      \begin{itemize}
		      \item Low temperature isotherms intersect with the blue region at the vapor pressure for that temperature
		      \item $31.1^{\circ}C$ is the \emph{critical} isotherm, the warmest isotherm which can exhibit liquid and gas co-existing in equilibrium
		      \item Supercritical fluids are shown above the critical pressure and critical temperature
	      \end{itemize}
	\item The \emph{compression factor ($Z$)} is a convenient way to characterize a gases deviation from perfect gas behavior
	      \begin{itemize}
		      \item $Z=\dfrac{V_m}{V_m^\circ}$
		      \item $pV_m = RTZ$
		      \item Figure 1C.3 shows hows how $Z$ varies with pressure for different gases
		      \item This behavior ($Z>1$ vs $Z<1$) can be explained by the relative importance of the repulsive and attractive portions of the potential curve
		      \item Table 1C.2 shows how different gases have a similar compression factor at the critical point, even if their critical constants vary widely
	      \end{itemize}
\end{itemize}
\paragraph{The Virial Equation}
\begin{itemize}
	\item A mathematical model for the compression factor could give us a reliable equation of state
	\item One approach is to expand $Z$ as a power series with respect to one of the state variables:
	      \begin{itemize}
		      \item $pV_m = RT\left(1+B^\prime p + C^\prime p^2 + \cdots\right)$
		      \item $pV_m = RT\left(1+\dfrac{B}{V_m} + \dfrac{C}{V_m^2}+\cdots\right)$ (More Common)
	      \end{itemize}
	\item These are different forms of the \emph{Virial Equation}
	\item The expansion could in principle be expanded indefinitely but in practice is limited to only the $B$ or sometimes also the $C$ terms
	\item Table 1C.1 gives the $B$ virial coefficient for several gases and two temperatures
	\item The \emph{Boyle temperature} is the temperature at which the second Virial coefficient ($B$) is zero
	\item This temperature exhibits behavior closest to that of a perfect gas over the widest ranges of pressure
\end{itemize}
\paragraph{The Van der Waals Equation of State}
\begin{itemize}
	\item The Virial equation makes sense mathematically, but is not easily related to specific properties of real gases
	\item Van der Waals (in 1873) proposed an equation which explicitly considered both the attractive forces and finite volume of real gas particles
	\item The van der Waals equation: $p=\dfrac{nRT}{V-nb}-a\dfrac{n^2}{V^2}$
	\item In this equation, $b$ accounts for the size of the gas particles, and $a$ accounts for the attractive forces
	\item To be clear, $a$ and $b$ are strictly empirical constants, but they very closely map on to these physical properties
	\item Table 1C.3 shows van der Waals coefficients for a variety of gases
	\item There are other equations of state, each with strengths and weaknesses
\end{itemize}
\paragraph{The Principle of Corresponding States}
\begin{itemize}
	\item The compression factor of different gases is similar at the critical point
	\item Using reduced state variables, gases give similar compression factors at all states
	      \begin{itemize}
		      \item $V_r = \dfrac{V_m}{V_c}$
		      \item $p_r=\dfrac{p}{p_c}$
		      \item $T_r=\dfrac{T}{T_c}$
	      \end{itemize}
	\item The van der Waals equation can be expressed in terms of reduced variables, and represent all gases simultaneously
	\item Reduced van der Waals equation: $p_r = \dfrac{8T_r}{3V_r-1}-\dfrac{3}{V_r^2}$
\end{itemize}
\paragraph{Table 1C.4 shows a variety of different equations of state}

\chapter{The First Law}
\begin{itemize}
	\item The first law of thermodynamics states that energy can be neither created nor destroyed, but only change form or transfer between systems
\end{itemize}
\section{Internal Energy}
\paragraph{Terms and Definitions}
\begin{itemize}
	\item When considering thermodynamic potentials, the universe is split into \emph{system} and \emph{surroundings}
	\item We must further define the interactions which can occur between the system and surroundings
	      \begin{description}
		      \item [Open System:] Energy and matter can transfer between system and surroundings
		      \item [Closed System:] Only energy can transfer between system and surroundings
		      \item [Isolated System:] Neither energy nor matter can transfer between system and surroudnings
	      \end{description}
	\item Energy can transfer as either heat or work, so boundaries can come in two different varieties as well
	      \begin{description}
		      \item [Diathermic:] Boundaries which can transfer heat
		      \item [Adiabatic:] Boundaries which do not transfew heat
	      \end{description}
	\item And lastly, processes come in two varieties
	      \begin{description}
		      \item[Exothermic:] A process which releases heat into the surroundings
		      \item[Endothermic:] A process which takes heat in from the surroundings
	      \end{description}
\end{itemize}
\paragraph{Developing the First Law}
\begin{itemize}
	\item \emph{Internal Energy} is the total energy of a system
	\item Internal energy is a state function, which depends only on state variables of a system and not on its past processes or the state of the surroundings
	\item Internal energy can be reported either as a total (extensive property) or molar (intensive property) amount
	\item Generally, internal energy can include both kinetic and potential energy
	\item The \emph{equipartition theory} states that at equilibrium the kinetic energy will be shared equally between all degrees of freedom
	\item Each degree of freedom will hold $\dfrac{1}{2}RT$ units of energy per mole
	\item For a monoatomic perfect gas, internal energy can be easily analyzed
	      \begin{itemize}
		      \item With no interactions, the perfect gas has no potential energy
		      \item There are also no vibrations or rotations, so tranlation is the only component of kinetic energy
		      \item The average translational kinetic energy in an ensemble of gas molecules is: $\dfrac{3}{2}k_BT$ for an individual molecule, or $\dfrac{3}{2}RT$ per mole
		      \item Note that the internal energy is independent of both $p$ and $V$
	      \end{itemize}
	\item For a linear diatomic at room temperature (2 rotations added), $U_m = \dfrac{5}{2}RT$
	\item For a non-linear molecule at room temperature (3 rotations added), $U_m = 3RT$
	\item Mathematically, the first law can be stated as: $\Delta U = q + w$
	\item Considering infinitesimal changes gives: $\mathrm{d}U = \mathrm{d}q + \mathrm{d}w$
\end{itemize}
\paragraph{Work}
\begin{itemize}
	\item Generally, we can express work as force times distance: $\mathrm{d}w = -\left|F\right|\mathrm{d}z$
	\item Table 2A.1 shows the equations for several forms of work
	\item For gas expansion work this becomes: $\mathrm{d}w = -p_{ex}\mathrm{d}V$
	\item The total work is: $w=-\displaystyle\int_{V_i}^{V_f}p_{ex}\mathrm{d}V$
	\item It is important to note that the pressure here is the \emph{external} pressure. This is true whether $\Delta V<0$ or $\Delta V>0$
	\item Indicator diagrams ($p$ vs $V$) are useful when analyzing changes in gas states
	\item Free expansions have no external pressure, and so involve no work
	\item Infinitesimal changes in the state variables make any change \emph{reversible}, but mostly we are concerned with two particular reversible processes
	\item Isothermal Reversible Expansion:
	      \begin{itemize}
		      \item For the process to be reversible, $p_{ex}=p$
		      \item This allows us to replace $w=-\displaystyle\int_{V_i}^{V_f}\!p_{ex}\mathrm{d}V$ with $w=-nRT\displaystyle\int_{V_i}^{V_f}\dfrac{\mathrm{d}V}{V} = -nRT\ln\dfrac{V_f}{V_i}$
	      \end{itemize}
	\item Adiabatic Reversible Expansion: These are a little more complex to analyze and are covered in section 2E
\end{itemize}
\paragraph{Heat}
\begin{itemize}
	\item On the molecular level, work can be seen as coordinate motion of the gas particles. Heat, on the other hand, represents a change in the velocities of \emph{random}, undirected particle motion
	\item In the absence of work (isochoric gas processes), $\displaystyle\int_i^f\mathrm{d}U = \displaystyle\int_i^f\mathrm{d}q_V$
	\item Constant volume calorimeters (``bomb calorimeters'') take advantage of this fact to relate internal energy changes to measured temperature changes
	\item For a bomb calorimeter: $q=-C\Delta T$, where $C$ is a calibrated heat capacity
	\item Heat capacity at constant volume is defined as $C_V=\left(\dfrac{\partial U}{\partial T}\right)_V$
	\item For a monoatomic perfect gas this gives: $C_{V,m}=\dfrac{\partial}{\partial T}\left(\dfrac{3}{2}RT\right)_V = \dfrac{3}{2}R = 12.47\dfrac{J}{mol~K}$
	\item And the heat at constant volume is: $q_V=C_V\Delta T = \Delta U_V$
\end{itemize}

\section{Enthalpy}
\begin{itemize}
	\item When the volume is no longer constant, internal energy is no longer easy to measure or useful in determining spontaneity
	\item Under constant pressure conditions, \emph{enthalpy} takes the place of internal energy in both respects
	\item Enthalpy is also a state function, and makes our second \emph{Thermodynamic Potential}
	\item $H = U + pV$
	\item $\mathrm{d}H = \mathrm{d}U + \mathrm{d}(pV)$ can be simplified in constant pressure to $\mathrm{d}H = \mathrm{d}U + p\mathrm{d}V$
	\item Since $\mathrm{d}U=\mathrm{d}q+\mathrm{d}w$, if only expansion work is involved then it can be further simplified to $\mathrm{d}H = \mathrm{d}q_p$
	\item Enthalpy can be measured with an \emph{isobaric} (coffee-cup) calorimeter
	\item $\Delta H = -C\Delta T$, where $C$ is a calibrated heat capacity
	\item $\Delta H$ can also be related to $\Delta U$
	      \begin{itemize}
		      \item For solids and liquids, $p\Delta V_m$ is often very small so $\Delta H \approx \Delta U$
		      \item For gases, $\Delta H \approx \Delta U + \Delta n_{gas}RT$
	      \end{itemize}
	\item Constant pressure heat capacity: $C_p = \left(\dfrac{\partial H}{\partial T}\right)_p$ and $\Delta H_p = C_p\Delta T$
	\item Isobaric heat capacity is a function of temperature, as different energetic degrees of freedom become available at higher temperature
	\item An common empirical equation for $C_p$ is: $C_{p,m} = a + bT + \dfrac{c}{T^2}$
	\item For a perfect gas, $C_p=C_V+nR$ and $C_{p,m}=C_{V,m}+R$
\end{itemize}

\section{Thermochemistry}
\paragraph{Standard Enthalpy Changes}
\begin{itemize}
	\item Studying the flow of energy through the course of a reaction is called thermochemistry
	\item Values in thermochemistry are often referenced to a \emph{standard state}, defined as $p=1~bar$ for gases
	\item Other phases, mixtures, etc. have different standard state definitions, but they are all notated with the symbol: $^{\std}$
	\item For example, the standard enthalpy of vaporization for water would be:

	      $\ch{H2O(l) -> H2O(g)} \hspace{2em} \Delta_{vap}H^{\std}(373~K) = 40.55~\nicefrac{kJ}{mol}$
	\item Because enthalpy is a state function, Hess's law applies
	\item e.g. $\Delta_{sub}H^{\std} = \Delta_{fus}H^{\std} + \Delta_{vap}H^{\std}$
	\item For a reaction, standard enthalpy change can be generally calculated as:

	      $\Delta H^{\std} = \displaystyle\sum\limits_{Products}\nu H^{\std}_m - \displaystyle\sum\limits_{Reactants} \nu H^{\std}_m$
	\item Usually, however, finding molar enthalpies is not as convenient as using enthalpies related to some defined elemental state (enthalpies of formation)
\end{itemize}
\paragraph{Standard Enthalpies of Formation}
\begin{itemize}
	\item For convenience, we define a reference state as an element at its most stable state and a pressure of $1~bar$
	\item Compounds can be formed from that reference state, with a enthalpy of reaction called the standard enthalpy of formation: $\Delta_fH^{\std}$
	\item One special exception is that for ionic solutions the hydrogen ion is also defined as a reference state ($\Delta_f H^{\std}(\ch{H^+}, aq) =0$)
	\item We can imagine a reaction as breaking the reactants apart into their elements, then using them to form the products
	\item Because enthalpy is a state function, it doesn't matter that this process is completely hypothetical
	\item $\Delta_rH^{\std}=\displaystyle\sum\limits_{Products}\nu \Delta_fH^{\std} - \displaystyle\sum\limits_{Reactants}\nu\Delta_fH^{\std}$
\end{itemize}
\paragraph{Temperature Dependence of Reaction Enthalpies}
\begin{itemize}
	\item When we have dealt with reaction enthalpies before, we have assumed that they are independent of temperature (e.g. $\Delta G = \Delta H-T\Delta S$)
	\item The enthalpies of reactants and products will both increase with temperature, but not at the same rate
	\item $C_{p}$ describes how the enthalpy of a substance changes with temperature:
	      $H(T_2)=H(T_1)+\displaystyle\int_{T_1}^{T_2}C_p\mathrm{d}T$
	\item Figure 2C.2 illustrates how the reaction enthalpy might change with temperature if $C_p$ is different for reactants and products
	\item This relation is summarized in Kirchoff's Law: $\Delta_rH^{\std}(T_2)=\Delta_rH^{\std}(T_1)+\displaystyle\int_{T_1}^{T_2}\Delta_rC_p^{\std}\mathrm{d}T$
	\item Note that this is the total change in heat capacity, taking into account the reaction stoichiometry:

	      $\Delta_rC_p^{\std}=\displaystyle\sum\limits_{Products}\nu C_p^{\std} - \displaystyle\sum\limits_{Reactants}\nu C_p^{\std}$
	\item The combustion fo hydrogen gas has $\Delta_{rxn}H^{\std}(298~K)=-241.82~\nicefrac{kJ}{mol}$
	      \begin{itemize}
		      \item Find $\Delta_{rxn}H^{\std}(273~K)$ $\left(-241.3~\nicefrac{kJ}{mol}\right)$
		      \item Find $\Delta_{rxn}H^{\std}(2527~K)$ (Low-T $C_p$ limit: $-288.15~\nicefrac{kJ}{mol}$ High-T $C_p$ limit: $-260.35~\nicefrac{kJ}{mol}$)
	      \end{itemize}
	\item Even this approach is an approximation, as the heat capacities of individual substances can change with temperature, as illustrated by the above example
\end{itemize}

\section{State Functions and Exact Differentials}
\begin{itemize}
	\item Changes in state functions can be found without dependance on the path from initial to final states
	\item This is why the differential of a state function is called and \emph{exact differential}: $\Delta U = \displaystyle\int_i^f\mathrm{d}U$
	\item Other functions depend on the path and are called path functions
	\item The differential of a path function is called an \emph{inexact differential}: $q=\displaystyle\int_i^f\dbar q$
	\item The ``$\dbar$'' indicates that this is an inexact differential and a path integral
\end{itemize}
\paragraph{Changes in Internal Energy}
\begin{itemize}
	\item Because internal energy is a function of both $V$ and $T$: $\mathrm{d}U = \left(\dfrac{\partial U}{\partial V}\right)_T\mathrm{d}V + \left(\dfrac{\partial U}{\partial T}\right)_V\mathrm{d}T$
	\item Note the natural variables of $U$, and reference the Wikipedia ``Thermodynamic Potentials'' page
	\item Partial derivatives of natural variables yield state variables, but here we differentiate with respect to easily measurable $V$ and $T$
	\item Sometimes a partial derivative has a straightforward physical interpretation (such as $C_p$ and $C_V$), and sometimes not
	\item The \emph{internal pressure} characterizes how internal energy changes with volume: $\pi_T = \left(\dfrac{\partial U}{\partial V}\right)_T$
	\item Figure 2D.4 illustrates how internal pressure can change with pressure
	\item So, the exact differntial is: $\mathrm{d}U = \pi_T\mathrm{d}V + C_V\mathrm{d}T$
	\item Figure 2D.2 illustrates what the exact and partial derivative mean
	\item Note that while we use $C_V$ in this expression, the differential is exact regardless of the conditions as long as both terms are included
	\item At constant pressure we can get new equations which give us new physical constants:
	\item The expansion coefficient: $\alpha = \dfrac{1}{V}\left(\dfrac{\partial V}{\partial T}\right)_p$
	\item The isothermal compressibility: $\kappa_T = -\dfrac{1}{V}\left(\dfrac{\partial V}{\partial p}\right)_T$
	\item With some algebra, we finally arrive at: $\left(\dfrac{\partial U}{\partial T}\right)_p = \alpha\pi_T V + C_V$
	\item And finally, we can (page 63) get the general expression: $C_p-C_V = \dfrac{\alpha^2 TV}{\kappa_T}$
\end{itemize}
\paragraph{The Joule-Thompson Effect}
\begin{itemize}
	\item We can examine the exact differential for enthalpy in a similar way:

	      $\mathrm{d}H = \left(\dfrac{\partial H}{\partial p}\right)_T\mathrm{d}p + \left(\dfrac{\partial H}{\partial T}\right)_p\mathrm{d}T$
	\item Use the isenthalpic condition to relate the two partial derivatives to each other, then re-write the exact differential as:

	      $\mathrm{d}H = -\mu C_p\mathrm{d}p + C_p\mathrm{d}T$
	\item $\mu$ is the Joule-Thompson coefficient, $\mu = \left(\dfrac{\partial T}{\partial p}\right)_H$
	\item $\mu$ can be directly observed under \emph{isenthalpic} conditions, as we will do next week in lab
	\item $\mu > 0$ for most gases under standard conditions, but can become negative under certain conditions
	\item Figures 2D.8 and 2D.9 show the heating and cooling regions in $T$/$p$ space
	\item The Joule-Thompson effect can be interpreted with respect to the exchange between potential and kinetic energy as gas molecules move farther apart in an expansion
\end{itemize}

\section{Adiabatic Changes}
\begin{itemize}
	\item We can find $\Delta U$ through a hypothetical two-step process shown in Figure 2F.1
	      \begin{itemize}
		      \item First an isothermal expansion. For an ideal gas this process keeps $U$ constant
		      \item Then an isochoric cooling. For any substance $\Delta U = C_V\Delta T$
	      \end{itemize}
	\item For an adiabatic process, there is no heat exhanged so $\delta U = w = C_V\Delta T$
	\item The temperature change can be found by: $V_iT_i^c = V_fT_f^c$ where $c=\dfrac{C_{V,m}}{R}$
	\item And the change in pressure is: $p_iV_i^\gamma=p_fV_f^\gamma$ where $\gamma = \dfrac{C_{p,m}}{C_{V,m}}$
\end{itemize}

\chapter{The Second and Third Laws}
\section{Entropy}
\paragraph{Discovering and Defining Entropy}
\begin{itemize}
	\item The direction of spontaneous change is \emph{not} always in the direction of lower energy
	\item Instead it relates to the distribution of energy, but understanding this concept will require the introduction f a new state variable, $S$
	\item You probably conceptualize $S$ as a measure of disorder. This isn't wrong, but it is very different from how early physicists thought when discovering the second law
	\item Consider a bouncing ball (Figures 3A.1 and 3A.2):
	      \begin{itemize}
		      \item Directed kinetic energy is converted into random thermal motion in the floor and air through friction and inelastic collisions
		      \item This process distributes the energy more widely, into more particles and more diverse energy modes
		      \item We wouldn't expect the random thermal motion of the floor to spontaneously transfer into a ball-at-rest as directed kinetic energy (i.e. the ball won't spontaneously bounce up from the floor)
		      \item Such a spontaneous coordination of the thermal energy would result in the floor getting colder, effectively converting thermal energy into work
	      \end{itemize}
	\item Kelvin's formulation of the second law (Figure 3A.3):

	      No process is possible in which the sole result is the absorption of heat from a reservoir and its complete conversion into work.

	\item This is actually synonymous with Clausius's formulation fo the second law (Figure 3A.4):

	      Heat does not flow spontaneously from a cool body to a hotter body.
	\item To transfer heat in this way will always require an amount of work
	\item The thermodynamic definition of entropy is: $\mathrm{d}S = \dfrac{\dbar q_{rev}}{T}$
	\item This definition came from studying heat engine cycles, and we will revisit it later, but for now suppose that it is true
	\item It is \emph{essential} that you use the heat from an equivalent reversible process if the actual process is irreversible
	\item For finding entropy change of the surroundings, you can use the actual heat since the surrounding are modeled as an infinite thermal reservoir
	\item The statistical definition of entropy is: $S=k_B\ln\mathcal{W}$
	\item $\mathcal{W}$ is the number of microstates available at a given temperature
	\item Hard spheres gas simulation
\end{itemize}
\paragraph{Entropy as a State Function -- Heat Engines}
\begin{itemize}
	\item If entropy is truly a state function, then $\displaystyle\oint\mathrm{d}S = \displaystyle\oint\dfrac{\dbar q_{rev}}{T}=0$
	\item Any arbitrary cycle can be approximated as a series of adiabats and isotherms (Figure 3A.10), so we need only consider those reversible processes
	\item These reversible processes can combine in a heat engine cycle known as a \emph{Carnot Cycle} (Figure 3A.7):
	      \begin{itemize}
		      \item Walk through the heat engine process qualitatively
		      \item When considering the entropy of each step, the adiabats don't contribute because $q=0$
		      \item The two isotherms take place at different temperatures, with heat entering the system during the isothermal expansion and leaving the system during the isothermal compression
		      \item $\displaystyle\oint\mathrm{d}S = \dfrac{q_h}{T_h} + \dfrac{q_c}{T_c}$
		      \item Assuming that the heat engine utilizes a perfect gas, we can prove the relation: $\dfrac{q_h}{q_c} = -\dfrac{T_h}{T_c}$
		      \item This proves that $\displaystyle\oint\mathrm{d}S = 0$ and that entropy is therefore a state function
	      \end{itemize}
	\item The next step is to prove that this result applies for any substance and not only for perfect gases
	\item Define the engine efficiency: $\eta = \dfrac{\mathrm{work~performed}}{\mathrm{heat~absorbed~from~hot~source}} = \dfrac{|w|}{|q_h|}$
	\item Energy conservation dictates that $|w| = |q_h|-|q_c|$, so we can re-write the efficiency as: $\eta = 1 - \dfrac{T_c}{T_h}$
	\item Now the efficiency of all reversible heat engines are equal, and greater than any non-reversible heat engines
	\item This can be shown by using a heat engine to power a heat pump. If the efficiencies were different, then they could be configured to convert heat directly into work, contrary to the second law
	\item This leads to the conclusion that the limits on efficiency apply regardless of configuration or working substance
	\item By comparing the efficiency of a carnot heat engine with different temperatures, absolute zero can be extrapolated

	      Let the hot source remain fixed, while decreasing the temperature of the cold sink, and extrapolate to the temperature which gives 100\% efficiency
	\item Ultimately, we can give the Clausius inequality: $\mathrm{d}S\geq\dfrac{\dbar q}{T}$
	\item The equality holds only for reversible processes
\end{itemize}
\paragraph{Calculating Entropy Changes for Various Processes}
\begin{itemize}
	\item Reversible Processes:
	      \begin{itemize}
		      \item For an isothermal process (on a perfect gas), $q = - w$ and $w = -nRT\ln\dfrac{V_f}{V_i}$, so $q = nRT\ln\dfrac{V_f}{V_i}$ and $\Delta S = nT\ln\dfrac{V_f}{V_i}$
		      \item For a reversible adiabatic process, $q=0$ so $\Delta S =0$
		      \item For a reversible isochoric process, $\dbar q=C_V\mathrm{T}$ so $\Delta S = \displaystyle\int\dfrac{\dbar q}{T} = \int\dfrac{C_V}{T}\mathrm{d}T$
		      \item If $C_V$ is independent of temperature over the relevant range it becomes $\Delta S = C_V\ln\dfrac{T_f}{T_i}$
		      \item Reversible isobaric warming similarly gives $\Delta S = \int\dfrac{C_V}{T}\mathrm{d}T = C_V\ln\dfrac{T_f}{T_i}$
		      \item For arbitrary processes, since $S$ is a state function we can model them as a combination of any two of the above
		      \item Phase transitions \emph{at the transition temperature} and constant pressure are: $\Delta_{trs}S=\dfrac{\Delta_{trs}H}{T_{trs}}$
		      \item For reversible processes $\Delta_{surr}S = -\Delta_{sys}S$
	      \end{itemize}
	\item Irreversible Processes:
	      \begin{itemize}
		      \item $\Delta_{surr}S = \displaystyle\int\dfrac{\dbar (-q_{sys})}{T_{surr}}$
		      \item For $\Delta S_{sys}$, model the change as reversible processes
		      \item For example, an irreversible adiabatic free expansion can be modeled as an isotherm
		      \item A phase change at a different temperature is modeled as an isobaric heating heating to $T_{trs}$, reversible phase change, and another isobaric heating to $T_f$
	      \end{itemize}
\end{itemize}
\section{The Measurement of Entropy}
\begin{itemize}
	\item One way to find the absolute molar entropy of a substance is the calorimetric measurement of entropy
	\item Essentially, you add up the integral of $\dfrac{C_p}{T}$ and the contribution from phase changes:

	      $S_m(T) = S_m(0) + \displaystyle\int_0^T\dfrac{C_{P,m}(T)}{T}\mathrm{d}T + \displaystyle\sum\limits_{transitions}\dfrac{\Delta_{trs}H}{T_{trs}}$
	\item Figure 3B.1 illustrates the integral in this expression
	\item Heat capacities for solids near $T=0$ are hard to directly measure, but can be modeled using the Debye approximation (extrapolating a cubic function)
	\item The third law of thermodynamics states that $S_m(0) = 0$ for all perfect crystalline substances
	\item We find $\Delta_{rxn}S$ entropies using Hess's law in the same way as $\Delta_{rxn} H$
\end{itemize}
\section{Concentrating on the System}
\begin{itemize}
	\item Re-arranging the Claussius inequality gives us: $\mathrm{d}S - \dfrac{\mathrm{d} q}{T} \geq 0$
	\item This is true for any spontaneous process, and it can be modified under specific conditions
	      \begin{itemize}
		      \item For constant volume, $\mathrm{d}S - \dfrac{\mathrm{d}U}{T}\geq0$ and $\mathrm{d}U-T\mathrm{d}S\leq0$
		      \item For constant pressure, $\mathrm{d}S - \dfrac{\mathrm{d}H}{T}\geq0$ and $\mathrm{d}H - T\mathrm{d}S \leq 0$
		      \item These expressions give a condition for spontanaety referring only to state variables!
	      \end{itemize}
	\item We can now define new thermodynamic potentials:
	      \begin{itemize}
		      \item Helmholtz energy: $A = U - TS$
		      \item Gibbs energy: $G = H-TS$
		      \item Conditions for spontanaety: $\mathrm{d}A_{T,V}\leq0$ and $\mathrm{d}G_{T,p}\leq0$
	      \end{itemize}
	\item The tendency toward spontanaety for exothermic reactions has more to do with increasing the entropy of the surroundings than with the system seeking lower energy
	\item The Helmholtz energy can also be shown to equal the maximum work possible for a process
	\item Similarly, the Gibbs energy is the maximum non-expansion work possible for a process
	\item In both cases, the maximum work is always less than the corresponding change in internal energy or enthalpy
	\item When $\Delta A=0$ or $\Delta G=0$ for processes in equilibrium under the appropriate conditions
	\item Like $H$ and $S$, we use tables of $\Delta_f G$ but must be careful because these values are only applicable at $T=298~K$
\end{itemize}
\section{Combining the First and Second Laws}
\begin{itemize}
	\item Recall that the exact differential of internal energy is: $\mathrm{d}U = T\mathrm{d}S - p\mathrm{d}V$
	\item This expresses $\mathrm{d}U$ in terms of only state variables, and so is path independent
	\item Mathematically, the exact differential of a state function gives rise to an important equality:
	      \begin{itemize}
		      \item For function $f(x,y)$, we can write $\mathrm{d}f = g\mathrm{d}x + h\mathrm{d}y$
		      \item It can also be shown to be true that: $\left(\dfrac{\partial g}{\partial y}\right)_x = \left(\dfrac{\partial h}{\partial x}\right)_y$
		      \item In the case of internal energy, this becomes: $\left(\dfrac{\partial T}{\partial V}\right)_S = -\left(\dfrac{\partial p}{\partial S}\right)_V$
		      \item Table 3D.1 shows all the Maxwell relations
	      \end{itemize}
\end{itemize}
\paragraph{Properties of the Gibbs Energy}
\begin{itemize}
	\item The differential form of Gibbs energy is: $\mathrm{d}G = V\mathrm{d}P-S\mathrm{d}T$ (From $G=H-TS$)
	\item This gives us a few differentials: $\left(\dfrac{\partial G}{\partial T}\right)_p = -S$ and $\left(\dfrac{\partial G}{\partial p}\right)_T = V$
	\item Figures 3D.2 and 3D.3 integrate these derivatives across $T$ and $p$
	\item The Gibbs-Helmholtz equation relates how entropy changes with temperature:

	      $\left(\dfrac{\partial \nicefrac{G}{T}}{\partial T}\right)_p = -\dfrac{H}{T^2}$
	\item This becomes particularly useful when relating energies of change: $\left(\dfrac{\partial \nicefrac{\Delta G}{T}}{\partial T}\right)_p = -\dfrac{\Delta H}{T^2}$
	\item Since the equilibrium composition depends on $G$, this equation tells how an equilibrium reaction might shift in response to a temperature change
	\item By examining $\Delta G = \Delta H - T\Delta S$, it might seem strange that enthalpy, and not entropy determines the shift with a temperature change. The Gibbs-Helmholtz equation explains the experiments which prove this to be the case
	\item As pressure increases, the Gibbs energy increases with a slope equal to the volume
	\item For a perfect gas, we can calculate the molar Gibbs energy at different pressures:

	      $G_m(p)=G_m^{\std}+RT\ln\dfrac{p}{p^{\std}}$
\end{itemize}
\paragraph{fugacity}
\begin{itemize}
	\item For real gases, the Gibbs molar energy deviates from the ideal
	\item We replace the pressure with a new effective pressure called the \emph{fugacity}:

	      $G_m(p)=G_m^{\std}+RT\ln\dfrac{f}{p^{\std}}$
	\item Here, $f = \phi p$ and $\ln\phi = \displaystyle\int_0^p \dfrac{Z-1}{p}\mathrm{d}p$
	\item Fugacity can be compared to Activity: $\mathcal{A} = \gamma C$
	\item The fugacity of all van der Waals gases follow the same curve if reduced variables are used (Figure 3D.8)
\end{itemize}

\chapter{Physical Transformations of Pure Substances}
\begin{itemize}
	\item A phase, generally, is any form of matter that is uniform throughout in chemical composition
	\item This goes beyond just solid, liquid, and gas
	\item Many substances have more than one solid phase, and solutions can have phases which vary by \% composition
	\item We call the number of components in a system $C$, and the number of phases in a system $P$
	\item One component can have multiple phases, and multiple components can exist in a single phase
	\item Phase transitions can be detected by thermal analysis (Figure 4A.2), even when the two phases are difficult to distinguish
	\item Here, thermodynamics and kinetics must be distinguished. Metastable phases persist even when a phase change becomes thermodynamically spontaneous
	\item Chemical potential ($\mu$) is a new, strange state variable, along with $n$
	      \begin{itemize}
		      \item For phase changes and reactions, we will eventually talk about how the number of particles of each species may change (think leCh\^{a}telier's principle)
		      \item Consider a chamber with a movable wall. We would determine how the volumes of the two sides might change by comparing their pressures
		      \item Similarly, when there is a possible transition (phase change, reaction, etc.), we determine how the $n$s might change by comparing the $\mu$s
		      \item At equilibrium, the chemical potential of all phases and species is the same
	      \end{itemize}
\end{itemize}
\section{Phase Diagrams of Pure Substances}
\begin{itemize}
	\item A phase diagram (Figure 4A.4) shows which phases are thermodynamically stable at various $p$s and $T$s
	\item At phase boundaries, multiple phases can coexist
	\item Condensed phases have a vapor pressure at all temperatures. The phase boundary occurs when $p_{vap}=p_{ext}$
	\item The phase rule describes the variance, or the number of variables which can be varied while maintaining equilibrium:

	      $F=C-P+2$ \hspace{3em} For pure substances, this is $F=3-P$
	\item So, 1-phase equilibria are areas, 2-phase equilibria are lines, and 3-phase equilibria are points
	\item The solid/liquid line for water is sloped negatively because of the low density of ice
	\item Discuss the phase diagrams in Figures 4A.7, 4A.9, and 4A.11
\end{itemize}

\section{Thermodynamic Aspects of Phase Transitions}
\begin{itemize}
	\item $\mu$, like $G$, changes with temperature according to molar entropy: $\left(\dfrac{\partial \mu}{\partial T}\right)_p=-S_m$
	\item Figure 4B.1 shows how the stability of different phases compete at different temperatures
	\item Applying pressure to chemical potential is again like $G$: $\left(\dfrac{\partial \mu}{\partial p}\right)_T=V_m$
	\item Figure 4B.2 shows how the a substance's freezing point will respond to changes in pressure
\end{itemize}
\paragraph{Phase Boundaries}
\begin{itemize}
	\item Phase boundaries occur where the chemical potentials of both phases are the same
	\item As both phases respond to changes in temperature and pressure, the line of the phase boundary keeps the potentials equal to each other
	\item The Clapeyron equation gives the slope of phase boundaries: $\dfrac{\mathrm{d}p}{\mathrm{d}T}=\dfrac{\Delta_{trs}S}{\Delta_{trs}V_m}$
	\item Since phase changes along the boundary are reversible, we can replace $\Delta_{trs}S=\dfrac{\Delta_{trs}H}{T}$, giving $\dfrac{\mathrm{d}p}{\mathrm{d}T}= \dfrac{\Delta_{trs}H}{T\Delta_{trs}V_m}$
	\item For vaporization, $\Delta_{vap}V_m = \dfrac{RT}{p}$, giving us $\dfrac{\mathrm{d}p}{\mathrm{d}T}= \dfrac{p\Delta_{vap}H}{RT^2}$
	\item Put the pressures on the same side, and use the identity $\dfrac{\mathrm{d}p}{p}=\mathrm{d}\ln p$
	\item This gives us the \emph{Clausius-Clapeyron equation}: $\dfrac{\mathrm{d}\ln p}{\mathrm{d}T} =\dfrac{\Delta_{vap}H}{RT^2}$
	\item Integrating across a temperature and assuming $\Delta_{vap} H$ is constant across the range gives:

	      $p_{T_2} = p_{T_1}e^{-\chi} \hspace{3em} \chi = \dfrac{\Delta_{vap}H}{R}\left(\dfrac{1}{T_2}-\dfrac{1}{T_1}\right)$
	\item For the solid/gas boundary, merely substitute the enthalpy of sublimation: $\Delta_{sub}H = \Delta_{fus}H+\Delta_{vap}H$
\end{itemize}
\paragraph{Ehrenfest Classification of Phase Transitions}
\begin{itemize}
	\item Phase transitions can be classified as first-order or second-order transitions
	\item All of the phase transitions familiar to you are first-order
	\item Figure 4B.9 shows the changes in $V_m$, $H$, $\mu$, $S$, and $C_p$ across phase transitions of both types
	\item A third type, $\lambda$ transitions, have $C_p$ approach $\infty$ at the transition, but does so via asymptotes rather than discontinuously
	\item Figure 4B.11 shows how a second-order transition might occur on a molecular level
\end{itemize}

\chapter{Simple Mixtures}
\section{The Thermodynamic Description of Mixtures}
\begin{itemize}
	\item In binary mixtures, the mole fractions are dependent according to: $\chi_A+\chi_B=1$
	\item The volume change as one component is added depends on the interactions between components
	\item Partial molar volume: $V_j=\left(\dfrac{\partial V}{\partial n_j}\right)_{p,t,n^\prime}$
	\item Figure 5A.1 shows the partial molar volumes of a water/ethanol mixture -- Note the min/max at $\chi\approx0.1$
	\item Adding $20~ml$ of ethanol to $20~ml$ water will give a final volume substantially lower than $40~ml$
	\item The total change in volume is:
\end{itemize}
\paragraph{The Chemical Potential}
\begin{itemize}
	\item The mathematical definition of chemical potential is: $\mu_j = \left(\dfrac{\partial G}{\partial n_j}\right)_{p,T,n^\prime}$
	\item This gives the exact differential of Gibbs energy as: $\mathrm{d}G = V\mathrm{d}p-S\mathrm{d}T+\mu_A\mathrm{d}n_A+\mu_B\mathrm{d}n_B+\ldots$
	\item At constant pressure and temperature, $\mathrm{d}G = \mu_A\mathrm{d}n_A+\mu_B\mathrm{d}n_B+\ldots$
	\item We can also state that the total Gibbs energy is: $G = n_A\mu_A + n_V\mu_B+\ldots$
	\item Now, this equation can be differentiated to give: $\mathrm{d}G = \mu_A\mathrm{d_A} + n_A\mathrm{d}\mu_A + \mu_B\mathrm{d}n_B + n_B\mathrm{d}\mu_B+\ldots$
	\item Setting this equal to the constant $T$ and $p$ expression lets us remove the $\mu\mathrm{d}n$ terms
	\item This gives the Gibbs-Duhem equation: $\displaystyle\sum n_j\mathrm{d}\mu_j = 0$
	\item The Gibbs-Duhem equation explicitly shows how the chemical potential of one component is dependent on the chemical potentials of all other components
\end{itemize}
\paragraph{Thermodynamics of Mixing}
\begin{itemize}
	\item The dependence of $\mu$ with $p$ is just like for $G$: $\mu=\mu^{\std}+RT\ln\dfrac{p}{p^{\std}}$
	\item We can use this to find the Gibbs energy of mixing for perfect gases
	      \begin{itemize}
		      \item Imagine a box split with one pure component on one side and another on the other
		      \item Mixing is analogous to expanding both components to fill the whole box (Figure 5A.6)
		      \item Because perfect gases don't interact, each gas expansion can be considered independently
		      \item $\Delta_{mix}G=\Delta \mu_A + \Delta \mu_B = n_A RT\ln\dfrac{p_A}{p} + n_B RT\ln\dfrac{p_B}{p}$
		      \item It is useful to simplify this and express it in terms of mole fractions:

		            $\Delta_{mix}G = nRT\left(\chi_A\ln\chi_A + \chi_B\ln\chi_B\right)$
	      \end{itemize}
	\item Entropy of mixing: $\Delta_{mix}S = -\left(\dfrac{\partial \Delta_{mix}G}{\partial T}\right)_{p,n_A,n_B} = -nR\left(\chi_A\ln\chi_A + \chi_B\ln\chi_B\right)$
	\item For a perfect gas, the enthalpy of mixing is $0$
\end{itemize}
\paragraph{Chemical Potentials of Liquids}
\begin{itemize}
	\item We can use the fact that two phases in equilibrium have the same chemical potential
	\item $\mu^\star_A = \mu^{\std}_A + RT\ln p^\star_A$ Where $\mu^\star_A$ is the chemical potential of a pure liquid
	\item This can be rearranged into $\mu_A^{\std} = \mu^\star_A-RT\ln p^\star_A$
	\item For a mixture, the liquid isn't pure: $\mu_A = \mu_A^{\std}+RT\ln p_A$
	\item Combining the two gives: $\mu_A = \mu_A^\star + RT\ln\dfrac{p_A}{p_A^\star}$
	\item Rault's law states that: $p_A = \chi_A p_A^{\star}$
	\item We finally get: $\mu_A = \mu_A^{\star} + RT\ln\chi_A$
	\item Rault's law works well for concentrated solutions, but dilute solutions follow Henry's law:

	      $p_B = \chi_B K_B$
	\item Figures 5A.12-14 Show vapor pressures as a function of $\chi$
\end{itemize}
\section{The Properties of Solutions}
\paragraph{Colligative Properties}
\begin{itemize}
	\item Colligative properties depend only on the mole fraction of a substance, but not its identity
	\item Figure 5B.6 shows how dissolved solutes can affect the freezing and boiling points of a substance
	\item Boiling point elevation: $\Delta T_b = K_b\chi_B$ Where $K_b=\dfrac{RT^{\star2}}{\Delta_{vap}H}$
	\item Freezing point depression: $\Delta T_f = K_f\chi_B$ Where $K_f=\dfrac{RT^{\star2}}{\Delta_{freeze}H}$
	\item Table 5B.1 gives $K_b$ and $K_f$ for some common substances
	\item Solubility can be estimated, recognizing that when solute \ch{B} is saturated in solvent \ch{A}: $\mu_{A}(l) = \mu_{{A}}(g)=\mu^\star_B(s)=\mu^\star_B+RT\ln\chi_B$
	\item Osmotic pressure is the pressure of solvent across a semi-permeable membrane with a concentration gradient
	\item Osmotic pressure follows an equation just like the perfect gas law: $\Pi = \dfrac{n_BRT}{V}$
\end{itemize}
\section{Phase Diagrams of Binary Systems}
\begin{itemize}
	\item For a mixture of two substances, the total vapor pressure must include the vapor pressures for both components:

	      $p = p_A+p_B=\chi_Ap^\star_A+\chi_Bp^\star_B=p^\star_B+\left(p^\star_A-p^\star_B\right)\chi_A$
	\item Figure 5C.1 shows the vapor pressure of a liquid following Rault's law
	\item When the pressure is less than the vapor pressure of either liquid, both are able to vaporize (The one makes head-space for the other)
	\item The two-dimensional region of coe-existant phases follows the rule $F = C - P + 2$
	\item The mole-fractions of the vapor phase ($y_A$ and $y_B$) do not match the mole-fractions of the liquid phase
	\item p.203 -- Equation 5C.4 should be: $y_A = \dfrac{\chi_Ap^\star_A}{p^\star_B+\left(p^\star_A-p^\star_B\right)\chi_A} = 1-y_B$
	\item We can express the total vapor pressure in terms of its composition as well:

	      $p = \dfrac{p^\star_Ap_B^\star}{P_A^\star+\left(p^\star_B-p^\star_A\right)y_A}$
	\item Figure 5C.4 shows a mole-fraction plot, combining $\chi$ and $y$
	\item Figure 5C.5 shows how to draw tie-lines
	      \begin{itemize}
		      \item Composition of the liquid and vapor phases are found by the intersections with the composition curves
		      \item The lever rule gives the fraction in vapor and liquid phases: $n_\alpha l_\alpha=n_\beta l_\beta$
	      \end{itemize}
	\item Considering the system in Figure 5C.8, with $z_A = 0.4$ and $T=80^\circ C$, estimate:
	      \begin{itemize}
		      \item $\chi_A$ and $\chi_B$
		      \item $y_A$ and $y_B$
		      \item The fraction of moles in the vapor and liquid phases
	      \end{itemize}
\end{itemize}
\paragraph{Distillation and Azeotropes}
\begin{itemize}
	\item Because the composition of the vapor phase is enriched in one component, we can use this as a method of purification
	\item Condensing the vapor and doing a second, third, etc. distillation can give very pure results
	\item Figures 5C.10 and 5C.11 illustrate this process
	\item Azeotropes have more complex composition curves with either a maximum or minimum boiling point
	\item Distillations with azeotropes result in one pure phase, and one phase at the min/max composition
	\item Figure 5C.12 shows a maximum boiling point azeotrope
	\item Figure 5C.13 shows a minimum boiling point azeotrope
	\item For these systems, which phase will have mixed composition, and what will be the composition?
\end{itemize}
\paragraph{Liquid-Liquid Phase Diagrams}
\begin{itemize}
	\item Some liquids are only miscible in certain ratios and at certain temperatures
	\item Demo with isopropyl alcohol/water mixture and beads
	\item Figure 5C.15 shows one example
	\item Like before, use tie-lines to determine the composition of the two phases (this time both are liquid)
	\item There will be one or two \emph{critical} temperatures, which mark the extremes where phase separation occur
	\item Figure 5C.23 shows the water-nicotine phase diagram with upper and lower critical temperatures
	\item When the temperature increases to the vapor pressure, we add the vapor composition curves
	\item Figure 5C.24 is when the upper critical temperature is below the point where any vapor exists
	\item Figure 5C.25 shows when the vapor composition curves collide with the liquid-liquid composition curves
\end{itemize}
\paragraph{Liquid-Solid Phase Diagrams}
\begin{itemize}
	\item Figure 5C.27 shows a typical liquid-solid phase diagram
	\item For most compositions, a pure solid will begin to freeze out first
	\item The composition will change as the one component freezes, until the mixture reaches the \emph{eutectic} composition
	\item Then a eutectic solid freezes out
	\item Eutectic mixtures can have dramatically lower melting points than the pure substances themselves
	\item Lead/Tin solder or gold/silicon
\end{itemize}
\section{Phase Diagrams of Ternary Systems}
Skip this section.

\section{Activities}
\begin{itemize}
	\item Fugacity relates how a gas is more or less chemically active than its pressure might suggest
	\item For solutions, \emph{activity} plays the same role
	\item For a solvent, $\mu_A = \mu^\star_A + RT\ln a_A$ where $a_A=\dfrac{p_A}{p^\star_A}$
	\item Note that this expression is general, and does not necessarily follow Rault's law
	\item For a solute we get an expression referenced to Henry's law:

	      $\mu_B=\mu^{\std}+RT\ln a_B$ where $a_B=\dfrac{p_B}{K_B}$
	\item In both cases we can give an activity coefficient $\gamma$ where $a_A=\gamma_A\chi_A$
	\item The parameter $\xi$ compares the solvent-solute interactions to the pure substance interactions
	      \begin{description}
		      \item[$\xi=0$] Is an ideal mixture, where mixing is driven entirely by entropy
		      \item[$\xi<0$] Gives exothermic mixing. Solvent-solute interactions are favorable, mixing is always spontaneous, and vapor pressures are depressed
		      \item[$\xi>0$] Gives endothermic mixing. Solvent-solute interactions are unfavorable, mixing is favorable at high temperatures, and vapor pressures are elevated
	      \end{description}
	\item This parameter is used in the Margules equations: $\ln\gamma_A = \xi\chi_B^2$ and $\ln\gamma_B=\xi\chi_A^2$
	\item Figure 5E.2 shows the vapor pressure of one component at various values of $\xi$, and Figure 5E.3 shows the total pressure of mixtures with two values of $\xi$
\end{itemize}
\section{The Activities of Ions}
\begin{itemize}
	\item Activities for ions are much more complex than for neutral solutes
	\item While we could assign a $\gamma$ for each ion, in practice it is far easier to work with $\gamma_{\pm}$
	\item Debye-H\"uckel limiting law:
	      \begin{itemize}
		      \item The Debye-H\"uckel limiting law states that: $\log \gamma_\pm = -A\left|z_+z_-\right|\sqrt{I}$
		      \item Here, $A$ is a constant related to the solvent. $A=0.509$ for water at $25^\circ C$
		      \item $I$ is the ionic strength: $I = \dfrac{1}{2}\displaystyle\sum z_i^2\left(\dfrac{b_i}{b^{\std}}\right)$ where $b^{\std} = 1~molal$
		      \item Table 5F.2 shows how even modestly concentrated solutions deviate significantly from ideality
		      \item At higher ionic strengths, this law is less reliable
	      \end{itemize}
	\item Extended Debye-H\"uckel law:
	      \begin{itemize}
		      \item The extended Debye-H\"uckel law: $\log \gamma_\pm = -\dfrac{A\left|z_+z_-\right|\sqrt{I}}{1+B\sqrt{I}}$
		      \item The Davies Equation: $\log \gamma_\pm = -\dfrac{A\left|z_+z_-\right|\sqrt{I}}{1+B\sqrt{I}} + CI$
		      \item Here, $B$ and $C$ are both dimensionless, empirical quantities
		      \item $B$ can be interpreted as the closest approach of two ions
	      \end{itemize}
	\item At lower salt concentrations, $\gamma_\pm<1$ and sparingly soluble solutes can be ``salted in''
	\item At higher salt concentrations, $\gamma_\pm>1$ and even miscible solutes can be ``salted out,'' as in the water-isopropanol system
	\item Consider the solubility of a sparingly soluble salt in solutions of various ionic strengths:
	      \begin{itemize}
		      \item Solve iteratively for a pure water solution
		      \item Assume that $I$ is buffered by the other salt for a salt-buffered solution
		      \item $K_{SP} = 9.8\times10^{-9}$ for \ch{PbI2} and $K_{SP} = 1.9\times10^{-13}$ for \ch{Mn(OH)2}
	      \end{itemize}
\end{itemize}
\chapter{Chemical Equilibrium}
\begin{itemize}
	\item In a chemical system, the system will evolve until it minimizes the total Gibbs energy
	\item For an equilibrium reaction, this means consuming reactants and producing products, or vice-versa
	\item We can call the extent of a reaction $\xi$, which is the number of moles of reaction which have occurred
	\item The reaction Gibbs energy is $\Delta_{rxn}G=\left(\dfrac{\partial G}{\partial \xi}\right)_{p,T} = \displaystyle\sum\limits_{Products}\nu_i\mu_i - \displaystyle\sum\limits_{Reactants}\nu_j\mu_j$
	\item Figure 6A.1 shows how $\Delta_{r}G$ might vary as a function of $\xi$
	\item When $\Delta_{rxn}G=0$ the reaction is at the equilibrium composition and will remain in a dynamic equilibrium
	\item When $\Delta_{rxn}G<0$, the forward reaction is spontaneous and the reaction is called \emph{exergonic}
	\item When $\Delta_{rxn}G>0$, the reverse reaction is spontaneous and the reaction is called \emph{endergonic}
	\item An exergonic reaction can be made to do work, and an endergonic reaction can be made to proceed anyway if sufficient work is input
\end{itemize}
\section{The Equilibrium Constant}
\begin{itemize}
	\item Consider a gas-phase reaction: \ch{A(g)<->B(g)}
	\item $\Delta_{rxn}G = \mu_B-\mu_A = \left(\mu_B^{\std}+RT\ln p_B\right)-\left(\mu_A^{\std}+RT\ln p_A\right)=\Delta_{rxn}G^{\std} + RT\ln\dfrac{p_B}{p_A}$
	\item Note that if $\Delta_{rxn}G^{\std}<0$ the reaction is not necessarily spontaneous under any but standard conditions
	\item This ratio of partial pressures is called the reaction quotient: $Q=\dfrac{p_B}{p_A}$
	\item Generally, $Q=\prod\mathcal{A}_i^{\nu_i}$ where $\nu_i$ are the signed stoichiometric coefficients
	\item For any reaction $\Delta_{rxn}G = \Delta_{rxn}G^{\std} + RT\ln Q$
	\item When the system is at equilibrium, $\Delta_{rxn}G=0$, and $Q$ becomes $K$
	\item We can rearrange the above equation to give: $RT\ln K = -\Delta_{rxn}G^{\std}$
	\item In fact, the existence of equilibrium reactions is quite surprising:
	      \begin{itemize}
		      \item The Gibbs energy of the pure reactants or the pure products will be lower, so you should always be able to minimize the Gibbs energy by going to one extreme
		      \item What is missed is the Gibbs energy of mixing for mixed composition in the middle
		      \item Figure 6A.3 shows how the total system Gibbs energy would vary with $\xi$ both neglecting and considering the Gibbs energy of mixing
		      \item The presence of a minimum is only possible when mixing is considered
	      \end{itemize}
	\item The equilibrium constant can be expressed in terms of concentrations ($K_c$) or of pressure ($K_p$)
	\item $K_p=K_c\left(\dfrac{c^{\std}RT}{p^{\std}}\right)^{\Delta \nu}$
	\item Figure 6A.4 shows the molecular interpretation of equilibrium. It maximizes the distribution of energy
\end{itemize}

\section{The Response of Equilibria to the Conditions}
\begin{itemize}
	\item Any perturbation from equilibrium will cause a response, or change in the extent of reaction ($\xi$)
	\item This response will act to restore equilibrium, and will often be in direct opposition to the perturbation
\end{itemize}
\paragraph{The Response to Pressure}
\begin{itemize}
	\item This is really the response to the \emph{partial} pressures of the reacting gases (As by a change in volume)
	\item I.e., adding an inert gas to increase the total pressure will not affect the value of $K_{p}$
	\item Changes in the partial pressures will result in a shift based on $\Delta_{rxn}n_{g}$
	\item Increases in pressure will shift $\xi$ toward the side with fewer moles of gas
\end{itemize}
\paragraph{The Response to Temperature}
\begin{itemize}
	\item Unlike with $p$, changes in $T$ affect the actual value of $K$
	\item The direction of the shift depends on $\Delta_{rxn}H$
	\item As a rule of thumb, we can treat “heat” as either a product or a reactant
	\item The van't Hoff equation, derived from Gibbs energy equations, gives the change in $\ln K$:

	      $\dfrac{\mathrm{d}\ln K}{\mathrm{d}\nicefrac{1}{T}} = -\dfrac{\Delta_{rxn}H^{\std}}{R}$
	\item If $\ln K$ is plotted against $\nicefrac{1}{T}$, the slope of the curve will be: $-\dfrac{\Delta_{rxn}H^{\std}}{R}$
	\item Integrating the van't Hoff equation gives: $\ln K_2 - \ln K_1 = -\dfrac{\Delta_{rxn}H^{\std}}{R}\left(\dfrac{1}{T_2}-\dfrac{1}{T_1}\right)$
	\item Note that here we use $\Delta_{rxn}H^{\std}$ even though standard conditions are not necessarily at equilibrium. This is because we are relying on the earlier equation: $RT\ln K = -\Delta_{rxn}G^{\std}$
\end{itemize}

\section{Electrochemical Cells}
\begin{itemize}
	\item Electrochemical cell construction and operation:
	      \begin{itemize}
		      \item The cathode is the electrode where reduction occurs (cathode and reduction both begin with consonants)
		      \item The anode is the electrode where oxidation occurs (anode and oxidation both begin with vowels)
		      \item This is true for both electrolytic and voltaic cells
		      \item The salt bridge completes the circuit, with anions flowing toward the anode, and cations flowing toward the cathode
	      \end{itemize}
	\item The half-reactions can be notated like this: \ch{Zn(s)|ZnSO4(aq)||CuSO4(aq)|Cu(s)}
	\item For a cell at standard conditions, we can use the half-cell reduction potentials and

	      $E_{cell}=E_{r,~cathode}-E_{r,~anode}$ (no consideration of $\nu$ is necessary)
	\item Under non-standard conditions, we use the Nernst equation: $-\nu FE_{cell}=\Delta_{rxn}G$
	\item $F$ is Faraday's constant, and $\nu$ is the moles of electrons transferred in the reaction
	\item We can also relate $\Delta_{rxn}G$ to the reaction quotient, as seen above. Done here gives:

	      $E_{cell}=E_{cell}^{\std}-\dfrac{RT}{\nu F}\ln Q$
	\item This equation also demonstrates why potential is, in one sense, independent of $\nu$. Double the reaction to double $\nu$, and you will also square $Q$ which cancels out the effect and gives the same $E_{cell}$
	\item Similarly, we can apply the relation between $G$ and $K$ to get: $E_{cell}^{\std}=\dfrac{RT}{\nu F}\ln K$
	\item Since $\left(\dfrac{\partial G}{\partial T}\right)_p=-S$, we can also say that $\dfrac{\mathrm{d}E_{cell}^{\std}}{\mathrm{d}T} = \dfrac{\Delta_{rxn}S^{\std}}{\nu F}$
	\item And finally, we find that $\Delta_{rxn} H^{\std} = -\nu F\left(E_{cell}^{\std}-T\dfrac{\mathrm{d}E_{cell}^{\std}}{\mathrm{d}T}\right)$
	\item Consider an electrochemical cell constructed with a \ch{AgCl(s)/Ag(s) + Cl^{-}(aq)} electrode and a \ch{Zn(s)/Zn^{2+}(aq)}  electrode
	      \begin{itemize}
		      \item Under standard conditions, what will be the cell potential?
		      \item Under standard conditions, which electrode will be the cathode and which will be the anode?
		      \item Draw the cell with a salt bridge, and indicate how charge will flow through every stage of the circuit
		      \item How would the addition of \ch{NaCl} into the \ch{AgCl(s)/Ag(s) + Cl^{-}(aq)} half-cell affect the cell potential (note that \ch{NaCl} will not participate in direct redox chemistry at all)
		      \item If a cell begins under standard conditions and is run until it "dies," what will be the final concentrations of \ch{Cl-} and \ch{Zn^{2+}}?
	      \end{itemize}
\end{itemize}
\section{Electrode Potentials}
\begin{itemize}
	\item Electric potential must be referenced to some standard. For cells, we reference the reduction of hydrogen
	\item We could find reduction potentials by measuring half-reactions against a standard hydrogen electrode
	\item In practice, we can also chain known redox reactions together to get an unknown half-cell potential (kind of like Hess's law)
	\item For example, if you know the potential of the redox couples \ch{Fe(s)/Fe^{2+}(aq)} and \ch{Fe^{2+}(aq)/Fe^{3+}(aq)}, you can calculate the potential for \ch{Fe(s)/Fe^{3+}(aq)}
	\item In this instance, the number of electrons matters! $\nu_CE^{\std}(C) = \nu_AE^{\std}(A) + \nu_BE^{\std}(B)$
	\item So, $3E^{std}(\ch{Fe(s)/Fe^{3+}}) = 2E^{\std}(\ch{Fe(s)/Fe^{2+}(aq)}) + 1E^{\std}[\ch{Fe^{2+}(aq)/Fe^{3+}(aq)}]$
\end{itemize}
\setcounter{chapter}{15}
\chapter{Molecules in Motion}
\section{Transport in Gases}
\begin{itemize}
	\item When a gas is not homogeneous, there will be a net flow of gas particles
	\item The flow of heat and viscosity behave similarly, so all will have similar equations
	\item Fick's first law of diffusion: $J_{\mathrm{Matter}}=-D\dfrac{\mathrm{d}\mathcal{N}}{\mathrm{d}z}$
	\item $D$ is called the “Diffusion Coefficient”
	\item For heat: $J_{\mathrm{Energy}} = -\kappa\dfrac{\mathrm{d}T}{\mathrm{d}z}$
	\item $\kappa$ is the coefficient of thermal conductivity
	\item $J_{\mathrm{x-Momentum}} = -\eta\dfrac{\mathrm{d}v_x}{\mathrm{d}z}$
	\item $\eta$ is the coefficient of viscosity
	\item We can understand transport by considering the collision flux through a slice along the gradient (Figure 19A.3)
	      \begin{itemize}
		      \item Collision frequency depends on the number density, temperature, and mass of a gas
		      \item With different densities across a gradient, the number of collisions from the dilute side is lower than the number of collisions from the concentrated side
		      \item Of course, particles don't actually collide since the slice is not a real barrier
		      \item The collisions represent particles traveling across the slice
		      \item The diffusion coefficient can be expressed as: $D=\frac{1}{3}\lambda v_{mean}=\frac{1}{3}\left(\dfrac{k_BT}{\sigma p}\right)\left(\dfrac{8RT}{\pi M}\right)^{\nicefrac{1}{2}}$
		      \item Higher pressures give slower diffusion
		      \item Hotter temperatures give faster diffusion
		      \item Smaller particles (both mass and size) give faster diffusion
	      \end{itemize}
	\item Things are a bit different for thermal conductivity
	      \begin{itemize}
		      \item $\kappa = \frac{1}{3}\nu v_{mean}\lambda\mathcal{N}k_B = \frac{1}{3}\lambda v_{mean}[J]C_{V,~m} = \dfrac{\nu pD}{T}$
		      \item Here, $\nu = \frac{1}{2}N_{D.o.F}$ (Degrees of freedom divided by 2)
		      \item $[J]$ is the molar concentration of the carrier particles $J$
		      \item Thermal conductivity is \emph{independent} of pressure
		      \item The heat capacity scales the energy gradient for a given temperature gradient
	      \end{itemize}
	\item And for viscosity
	      \begin{itemize}
		      \item $\eta = \frac{1}{3}v_{mean}\lambda m\mathcal{N} = MD[J] = \dfrac{pMD}{RT}$
		      \item $m$ is the particle mass, $M$ is the molar mass, and $\lambda$ is the mean free path
		      \item $[J]$ is the molar concentration of the momentum-carrying particles
		      \item Viscosity is also independent of pressure
		      \item Hotter temperatures increase the viscosity of gases (This is the reverse of liquids, which must overcome intermolecular forces)
	      \end{itemize}
	\item Effusion is also a transport phenomenon
	      \begin{itemize}
		      \item Again, consider “collisions” - this time collisions with the hole
		      \item $Rate = Z_WA_0=\dfrac{pA_0}{\sqrt{2\pi m k_BT}}=\dfrac{pA_0N_A}{\sqrt{2\pi MRT}}$
		      \item $A_0$ is the area of the hole, and $Z_W$ is the collision rate
		      \item Note that $p$ is a function of $T$, so the final dependence is $Rate\propto\sqrt{T}$
		      \item $Rate\propto M^{-\nicefrac{1}{2}}$
	      \end{itemize}
\end{itemize}

\section{Motion in Liquids}
\begin{itemize}
	\item For liquids to flow, intermolecular forces must be overcome
	\item This gives a temperature dependence at constant volume of: $\eta=\eta_0e^{\nicefrac{E_a}{RT}}$
	\item Under constant pressure, the thermal expansion is a more important factor than the activation energy
\end{itemize}
\paragraph*{Conductivity in Electrolytes}
\begin{itemize}
	\item For any electrical conductor, the conductance $G$ is the inverse resistance: $G=\dfrac{1}{R}$
	\item Conductance can also be expressed in terms of conductivity ($\kappa$), cross-sectional area ($A$), and length ($l$)

	      $G=\kappa\dfrac{A}{l}$
	\item For a solution, we can expect that more ions would lead to higher conductivity
	\item This leads to the \emph{molar conductivity}: $\Lambda_m=\dfrac{\kappa}{c}$
	\item It is even worse, however, since the molar conductivity is not constant, but varies with concentration too: $\Lambda_m = \Lambda_m^\circ-\mathcal{K}\sqrt{c}$ ~ ($\mathcal{K}$ is an empirical constant)
	\item On the microscopic level, we see that electrolyte conductivity has to do with ion mobility
	      \begin{itemize}
		      \item Ions in solution are met with a coefficient of friction: $f=6\pi\eta a$
		      \item Here, $a$ is the hydrodynamic radius (or Stokes radius)
		      \item The ion mobility is: $u = \dfrac{ze}{f}$ ($z$ is the ion charge and $e$ is the elementary charge)
		      \item The ion drift speed is: $s = uE$
		      \item Molar ion conductivity is: $\lambda=zuF$
		      \item Table 19B.2 shows a few ion mobilities. Note the Grotthuss mechanism for \ch{H+} and \ch{OH-}
	      \end{itemize}
	\item Skip the Einstein relations, even though they are very cool
\end{itemize}

\section{Diffusion}
\begin{itemize}
	\item Fick's first law, $J_{\mathrm{Matter}}=-D\dfrac{\mathrm{d}\mathcal{N}}{\mathrm{d}z}$, only addresses an instantaneous flux
	\item Fick's second law, or the diffusion equation, tells how concentrations change over time:

	      $\dfrac{\partial c}{\partial t} = D\dfrac{\partial^2 c}{\partial x^2}$
	\item We see that the concentration over time will change according to the curvature of the concentration gradient
	\item Note that this is a partial derivative with respect to $x$. In 3-D space, you would have to include the partials along $y$ and $z$ as well
	\item Integrating Fick's second law gives the time-dependence of the concentration:

	      $c(x,t)=\dfrac{n_0}{A\sqrt{\pi Dt}}e^{-\nicefrac{x^2}{4Dt}}$ \hspace{2em} for 1 dimension

	      $c(r,t)=\dfrac{n_0}{8\left(\pi Dt\right)^{\nicefrac{3}{2}}}e^{-\nicefrac{r^2}{4Dt}}$ \hspace{2em} for 3 dimensions
	\item $A$ is the cross-sectional area and $n_0$ is the initial number of moles which start at point $x,r=0$
	\item We can quantify the extent of diffusion by the rms displacement: $x_{rms} = \sqrt{2Dt}$
	\item Generally, diffusion can be very slow. Mixtures must be well stirred for efficient reaction
\end{itemize}
\paragraph*{Practice and Application}
\begin{itemize}
	\item Gas diffusion
	      \begin{itemize}
		      \item YouTube videos on Bromine diffusion by channel Isaac Physics
		      \item Calculate the diffusion coefficient for Bromine in $1~atm$ and in $0.01~atm$ of pressure
		      \item Predict the $x_rms$ after $10$ minutes, and compare to the video
		      \item Use the diffusion spreadsheet
	      \end{itemize}
	\item Viscosity and electrical conductance
	      \begin{itemize}
		      \item A fluorescent lamp is filled with $0.003~atm$ of mercury vapor
		      \item Find $\eta$ for the mercury vapor at $315~K$
		      \item Assume that while the lamp is lit all mercury is in the \ch{Hg^{2+}} state
		      \item What is the ion mobility? (assume $a\approx1.00$\AA)
		      \item What is the ion drift speed at $120~V$?
		      \item What is the molar ion conductivity?
		      \item What is the resistance, and current through the lamp (assume \ch{Hg^{2+}} is the sole charge-carrier)?
		      \item Note that in reality the plasma in a fluorescent lamp has free electrons, unlike in an electrolyte solution. Those electrons actually carry the charge in these lamps
	      \end{itemize}
\end{itemize}

\chapter{Chemical Kinetics}
\section{The Rates of Chemical Reactions}
\begin{itemize}
	\item For a reaction: \ch{aA + bB = cD + dD}, $v=-\dfrac{\mathrm{d}[A]}{a\mathrm{d}t} = -\dfrac{\mathrm{d}[B]}{b\mathrm{d}t} = \dfrac{\mathrm{d}[C]}{c\mathrm{d}t} = \dfrac{\mathrm{d}[D]}{d\mathrm{d}t}$
	\item The extent of the reaction can be expressed as: $\xi=\dfrac{n_J-n_{J,0}}{\nu_J}$
	\item This gives the rate as: $v = \dfrac{1}{V}\dfrac{\mathrm{d}\xi}{\mathrm{d}t}$
	\item If the concentration of a species can be monitored over time (e.g. spectrophotometrically) then the tangent slope will give the rate at any given time (Figure 20A.3)
	\item The rate is ultimately governed by the opportunities for reactants to encounter and react together
	\item Therefore, the rate can also be expressed in terms of the reactant concentrations: $v=k_r[A]^m[B]^n$
	\item This form is called the rate law, and the exponents must be found experimentally
	\item The exponents give the reaction order with respect to each reactant, and overall
	\item The units of $k_r$ must correspond to the overall reaction order to give final units of $v=\nicefrac{mol}{ls}$
	\item Method of initial rates to determine the reaction order:
	      \begin{itemize}
		      \item The effect of a single reactant can be isolated by using all other reactants in large excess
		      \item This creates a psudo-reaction order that depends only on the limited reactant
		      \item A method must be used to determine the average initial rate $v_0$ (color indicator, spectrophotometrically, electrochemically, etc.)
		      \item Varying the concentration of the limiting reactant and comparing the rates can give the reaction order
		      \item $\log v_0 = \log k_r^\prime + a \log [A]_0$
		      \item This method will not reveal complex kinetics, such as product-catalyzed reactions
	      \end{itemize}
	\item Sample data for initial rate method with reaction \ch{A + B -> C}:
\end{itemize}
\begin{tabular}{c|c|c|c}
	Run & $[A]_0$   & $[B]_0$   & Rate $\nicefrac{M}{s}$ \\ \midrule \midrule
	1   & $1.0~M$   & $0.010~M$ & $1.5\times10^-4$       \\ \midrule
	2   & $1.0~M$   & $0.030~M$ & $4.5\times10^-4$       \\ \midrule
	3   & $0.010~M$ & $1.0~M$   & $1.5\times10^-6$       \\ \midrule
	4   & $0.030~M$ & $1.0~M$   & $1.4\times10^-5$       \\
\end{tabular}
\begin{itemize}
	\item Give the rate law, including $k_r$ with proper units
\end{itemize}
\section{Integrated Rate Laws}
\begin{itemize}
	\item We can equate the two forms of the rate to get: $\dfrac{\mathrm{d}[A]}{a\mathrm{d}t} = -k_r[A]^m$
	\item This differential equation can be solved for different cases of the exponent $m$
	\item First Order:
	      \begin{itemize}
		      \item If $m=1$, then the integrated form is: $\ln \dfrac{[A]}{[A]_0}=-k_rt \hspace{2em} [A]=[A]_0e^{-k_rt}$
		      \item If $\ln [A]$ is plotted against time, then the data will fit a line with $slope = -k_r$
		      \item We can also consider the special case where $[A] = \frac{1}{2}[A]_0$ and $t_{\nicefrac{1}{2}} = \dfrac{\ln 2}{k_r}$
		      \item For first-order reactions, there is a characteristic time constant: $\tau = \dfrac{1}{k_r}$
	      \end{itemize}
	\item Second-Order:
	      \begin{itemize}
		      \item If $m=2$, then the integrated form is: $\dfrac{1}{[A]}-\dfrac{1}{[A]_0} = k_rt \hspace{2em} [A] = \dfrac{[A]_0}{1+k_rt[A]_0}$
		      \item If $\dfrac{1}{[A]}$ is plotted against time, then the data will fit a line with $slope=k_r$
		      \item For 2nd-order reactions, the half-life is not constant: $t_{\nicefrac{1}{2}}=\dfrac{1}{k_r[A]_0}$
	      \end{itemize}
\end{itemize}

\section{Reactions Approaching Equilibrium}
\begin{itemize}
	\item As reactions approach dynamic equilibrium, the rate law must take into account the reverse reaction too
	\item For the reaction \ch{A->B}, the master rate equation is: $\dfrac{\mathrm{d}[A]}{\mathrm{d}t}=-k_r[A]+k_r^\prime[B]$
	\item At equilibrium, the forward and reverse rates will be the same, so $k_r[A]=k_r^\prime[B]$
	\item This means we can relate the rate constants to the equilibrium constant: $K=\dfrac{k_r}{k_r^\prime}$
	\item If the system is perturbed away from equilibrium (say, by a sudden increase in temperature or pressure) it will relax into the new equilibrium according to: $\chi = \chi_0e^{-\nicefrac{t}{\tau}}$ and $\tau = \dfrac{1}{k_r + k_r^\prime}$
\end{itemize}

\section{The Arrhenius Equation}
\begin{itemize}
	\item So far we have talked about $k_r$ as simply an empirical constant (found by fitting data)
	\item Arrhenius found the dependence of $k_r$ on temperature, and proposed a theory to describe this behavior
	\item $\ln k_r = \ln A - \dfrac{E_a}{RT}$ or $k_r = Ae^{-\frac{E_a}{RT}}$
	\item The activation energy can be plotting $\ln k_r$ vs $\nicefrac{1}{T}$: $slope = -\dfrac{E_a}{R}$
	\item $A$ is the pre-exponential factor, or frequency factor
	\item The frequency factor can be interpreted as the rate of “reaction attempts,” or collisions with the proper orientation for a reaction to occur
	\item A reaction coordinate diagram (Figure 20D.3) shows how a reaction might proceed
	      \begin{itemize}
		      \item The $E_a$ and $\Delta H_{rxn}$ are found on the diagram
		      \item The maximum energy point is called an activated complex or transition state
		      \item Catalysts change the pathway taken in the reaction coordinate diagram
		      \item What is the physical interpretation for the x-axis in this diagram?
	      \end{itemize}
\end{itemize}

\section{Reaction Mechanisms}
\begin{itemize}
	\item Many reactions actually take place in a series of steps
	\item Each step (and any one-step reactions) are called “elementary reactions”
	\item Elementary reactions can have different molecularity
	      \begin{itemize}
		      \item Molecularity is the number of molecules that come together for the elementary reaction
		      \item Molecularity is also like reaction order in the rate law for an elementary reaction
		      \item Unimolecular elementary reactions: \ch{A->P}\hspace{2em}$rate=k_r[A]$
		      \item Bimolecular elementary reactions: \ch{A+B->P}\hspace{2em}$rate=k_r[A][B]$
		      \item Trimolecular elementary reactions\ldots don't exist!
		      \item It is important to remember that the reaction order of an overall reaction is not necessarily related to the molecularity of the elementary steps
	      \end{itemize}
	\item Multi-step reactions:
	      \begin{itemize}
		      \item Multi-step reactions proceed through reaction intermediates
		      \item Intermediates are not the same as activated complexes (transition states)
		      \item Figure 20E.1 shows how the concentrations of
	      \end{itemize}
	\item The steady-state approximation:
	      \begin{itemize}
		      \item The steady-state approximation assumes that any intermediates maintain a low, steady concentration after an initial induction period
		      \item A steady state is distinct from equilibrium
		      \item This is the case when the second step is much faster than the first step ($k_b\gg k_a$)
		      \item Figure 20E.2 shows the reactant, product, and intermediate concentrations under this model
		      \item The concentration of the intermediate can be found by setting its master rate equation equal to 0
		      \item Solve the rate equation for the reaction \ch{A -> I -> B}
		      \item The overall rate law can be predicted by the “slow step”
		      \item Figure 20E.5 shows a reaction coordinate diagram for this case
	      \end{itemize}
	\item Pre-equilibria
	      \begin{itemize}
		      \item Consider the reaction: \ch{A+B<->I->P}
		      \item In this case, the second step is much slower than the first step, allowing equilibrium to be reached
		      \item Solve the rate law
	      \end{itemize}
	\item If more than one product can be formed, kinetics and thermodynamics are both relevant
	\item In the long time limit, the thermodynamic product is produced
	\item In the short time limit, the kinetic product is produced
\end{itemize}

\section{Examples of Reaction Mechanisms}
\begin{itemize}
	\item Lindemann-Hinshelwood mechanism
	      \begin{itemize}
		      \item Some gas-phase reactions exhibit first-order kinetics, but they must occur via bimolecular collisions
		      \item The Lindemann-Hinshelwood mechanism solves this question

		            \ch{A+A->A*+A}

		            \ch{A+A*->A+A}

		            \ch{A*->P}
		      \item $\dfrac{\mathrm{d}[P]}{\mathrm{d}t} = k_b[A^*]$ and  $\dfrac{\mathrm{d}[A^*]}{\mathrm{d}t} = k_a[A]^2-k^\prime_a[A][A^*]-k_b[A^*]$
		      \item $[A^*]=\dfrac{k_a[A]^2}{k_b+k^\prime_a[A]}$
		      \item $rate = \dfrac{k_ak_b[A]^2}{k_b+k^\prime_a[A]}$
		      \item This is only first-order if $k_a^\prime[A]\gg k_b$, so going to lower pressures reveals the more complex kinetics
	      \end{itemize}
	\item Skip polymerization except to say that we can talk statistically about how long an average polymer chain will be based on the reaction kinetics
\end{itemize}

\section{Photochemistry}
\begin{itemize}
	\item Photochemical processes are chemical reactions which require a photoexcited reactant (Table 20G.1)
	\item Photophysical processes are processes which involve changes in electronic states, but not the breaking or forming of bonds (Table 20G.2)
	\item Timescales are in femtoseconds for electronic transitions, picoseconds for fluorescence (excited state lifetime), and milliseconds to seconds for phosphorescence
	\item Quantum yield ($\phi$) is the fraction of absorption events which lead to the product or process of interest
	\item Excited state decay
	      \begin{itemize}
		      \item The rate of absorption is merely the intensity of absorption
		      \item Fluorescence ($k_F$), internal conversion ($k_{IC}$), and intersystem crossing ($k_{ISC}$) are unimolecular decay processes
		      \item $\tau=\dfrac{1}{k_F+k_{IC}+k_{ISC}}$
		      \item $\phi_F=\dfrac{k_F}{k_F+k_{IC}+k_{ISC}}$
	      \end{itemize}
	\item Quenching
	      \begin{itemize}
		      \item Any side-reactions which reduce the quantum yield contribute to quenching
		      \item By varying the concentration of quencher, we can find the rate of the quenching reaction
		      \item A quenching reaction is: \ch{S^* + Q -> S + Q} $\hspace{2em} v_Q=k_Q[Q][S^*]$
		      \item Stern-Volmer Equation: $\dfrac{\phi_{0}}{\phi} = 1 + \tau_0k_Q[Q]$
		      \item Self-quenching and the paradox: Aren't photoexcited molecules constantly bumping into solvent molecules?
		      \item The answer lies in the requirement for resonance in collisional energy transfer as well as for photon absorption
	      \end{itemize}
	\item F\"orster Theory
	      \begin{itemize}
		      \item Another quenching reaction is resonance energy transfer: \ch{S^* + Q -> S + Q^*}
		      \item Usually we are interested in the fluorescence of the acceptor molecule (Q)
		      \item Efficiency of transfer is: $\eta_T = 1 - \dfrac{\phi_F}{\phi_{F,0}}$
		      \item Transfer is efficient when:
		            \begin{itemize}
			            \item Donor and acceptor are able to get close to each other
			            \item Donor fluorescence and acceptor absorbance spectra overlap (resonance)
		            \end{itemize}
		      \item F\"orster theory describes the distance relationship: $\eta_T = \dfrac{R_0^6}{R_0^6 + R^6}$
		      \item $R_0$ is a parameter defined as the distance where transfer is $50\%$ efficient
		      \item FRET spectroscopy is used to measure molecular distances in proteins and other biomolecules
	      \end{itemize}
\end{itemize}

\section{Enzymes}
\begin{itemize}
	\item Enzymes are biomolecules which act as catalysts for essential biochemical reactions
	\item The substrates dock into the active sites of enzymes
	\item The “lock-and-key” model has been largely supplanted by the “induced fit” model
	\item The Michaelis-Menten Mechanism
	      \begin{itemize}
		      \item \ch{E + S <-> ES}
		      \item \ch{ES -> P + E}
		      \item The rate equation will be: $\dfrac{k_b[E]_0}{1 + \nicefrac{K_M}{[S]_0}}$
		      \item Here, $K_M = \dfrac{k_a^\prime+k_b}{k_a}$
		      \item Lineweaver-Burk Plots (Figure 20H.4)
		            \begin{itemize}
			            \item $\dfrac{1}{[S]}$ vs $\dfrac{1}{v}$
			            \item Slope is $\dfrac{K_M}{v_{max}}$, y-intercept is $\dfrac{1}{v_{max}}$, and x-intercept is $\dfrac{-1}{K_M}$
		            \end{itemize}
	      \end{itemize}
	\item The turnover rate is a measure of how quickly an enzyme can do its work: $k_{cat}=k_b=\dfrac{v_{max}}{[E]_0}$
	\item The catalytic efficiency can be interpreted as an effective rate constant, but it is also related to the fraction of substrate docking events which eventually lead to product: $\eta = \dfrac{k_{cat}}{K_M}=\dfrac{k_ak_b}{k_a^\prime+k_b}$
	\item Enzyme inhibition:
	      \begin{itemize}
		      \item Inhibition is when a different, inactive substrate can also bind with the enzyme and prevent or delay production of products

		            \ch{EIn <-> E + In}

		            \ch{ESIn <-> ES + In}
		      \item Don't worry about the complex rate equation for inhibited enzymes
		      \item Figure 20H.6 shows the effects of different types of inhibition
		      \item Competitive inhibition: An inhibitor binds to the active site, blocking access for the substrate
		      \item Uncompetitive inhibition: An inhibitor binds to some other site, but only if the substrate is already present at the active site, and prevents product formation
		      \item Non-competitive inhibition: An inhibitor binds to some other site and reduces the binding activity of the active site
	      \end{itemize}
\end{itemize}

\chapter{Reaction Dynamics}
\section{Collision Theory}
\begin{itemize}
	\item Here we will analyze the exponential pre-factor
	\item Recall from chapter 1 that the collision frequency is: $z=\sigma v_{rel}\mathcal{N}\hspace{2em}\mathcal{N} = \dfrac{N}{V} = \dfrac{pN_A}{RT} = \dfrac{p}{k_BT}$
	\item This number is the collision rate from the perspective of a single particle
	\item To find the rate of collisions throughout a substance, we use: $Z_{AB}=\sigma v_{rel}N_A^2[A][B]$
	\item For dissimilar molecules, $v_{rel}=\left(\dfrac{8kT}{\pi\mu}\right)^{\nicefrac{1}{2}}$
	\item Not all collisions lead to a reaction due to an activation energy requirement: $e^{-\frac{E_a}{RT}}$
	\item Not all collisions lead to a reaction due to a steric factor ($P$)
	      \begin{itemize}
		      \item The steric factor is usually less than $1$ because improper orientation will stop a reaction
		      \item The steric factor can be greater than $1$ if molecules need not actually collide to react
		      \item For example in the harpoon mechanism of \ch{K + Br2 -> KBr + Br}
		      \item The RRK model accounts for collisions which have enough energy, but distribute that energy into modes which are not conducive to reaction
		      \item RRK theory gives $P=\left(1-\dfrac{E^{\star}}{E}\right)^{s-1}$ for the Lindemann mechanism
		      \item $s$ is the number of energetic modes
		      \item $E$ is the kinetic energy transferred in the collision and $E^{\star}$ is the energy needed to sufficiently activate the reactive molecule
	      \end{itemize}
	\item Taken together, the rate is: $v=PZ_{AB}e^{-\frac{E_a}{RT}} = P\sigma v_{rel}N_A^2[A][B]e^{-\frac{E_a}{RT}}$
	\item This means that for collision theory, $k_r=P\sigma v_{rel}N_A^2e^{-\frac{E_a}{RT}}$
	\item $P\sigma$ can be interpreted to mean the reactive cross-section (rather than the van der Waals collisional cross-section alone). This is particularly appropriate for the harpoon mechanism
\end{itemize}

\section{Diffusion-Controlled Reactions}
\begin{itemize}
	\item In liquids, reactants and products encounter each other far less frequently, but the encounters last much longer
	\item This “cage effect” gives an encounter pair the opportunity to acquire enough energy to overcome the activation energy barrier, even if they didn't have it initially
	\item Many solvent reactions can be described by the following mechanism:

	      \ch{A + B -> AB} \hspace{2em} $v=k_d[A][B]$

	      \ch{AB -> A + B} \hspace{2em} $v=k^\prime_d[AB]$

	      \ch{AB -> P} \hspace{2em} $v=k_a[AB]$
	\item Applying a steady-state approximation gives $v=\dfrac{k_ak_d}{k_a+k_d^\prime}[A][B]$
	\item There are two extreme limits: Diffusion controlled and activation-controlled
	\item Diffusion controlled ($k^\prime_d\ll k_a$): $k_r\approx k_d$
	\item Activation controlled ($k_a\ll k^\prime_d$): $k_r\approx\dfrac{k_ak_d}{k^\prime_d}\approx k_aK$
	\item $k_d$ is the rate at which molecules diffuse together: $k_d=4\pi R^{\star}DN_A$
	\item Here, $D$ is the sum of the diffusion coefficients for both reactants and $R^{\star}$ is the reaction distance
	\item Using the Stokes-Einstein equation and a convenient approximation of $R^{\star}$ simplifies this to: $k_d=\dfrac{8RT}{3\eta}$
	\item Note that $R$ here is the gas constant, and this expression is now general for all reactions in a given solvent
\end{itemize}

\section{Transition-State Theory}
\begin{itemize}
	\item Transition state theory (or activated complex theory) attempts to refine our estimates of the value of $k_r$ based on statistical mechanics and the structure of the transition state
	\item Assume that there is an equilibrium established between reactants and the activated complex \ch{A+B<->C$^{\ddagger}$} with equilibrium constant $K^{\ddagger}$
	\item The activated complex can dissociate to products with rate constant $k^{\ddagger}$
	\item In the end, the effective rate constant is: $k_r=RTk^{\ddagger}K^{\ddagger}$
	      %	\item What makes this different from a normal 2-state mechanism is that \ch{C$^{\ddagger}$} is not a metastable intermediate, but an unstable transition state
	\item Around the transition state, bonds are forming and breaking with an oscillatory frequency of $\nu^{\ddagger}$
	\item $k^{\ddagger}=\kappa\nu^{\ddagger}$ where $\kappa$ is assumed to be 1
	\item We will cover vibrational structure, etc. next year, but
\end{itemize}
\section{The Dynamics of Molecular Collisions}

\section{Electron Transfer in Homogeneous Systems}

\section{Processes at Electrodes}

\chapter{Processes on Solid Surfaces}
\section{An Introduction to Solid Surfaces}

\section{Adsorption and Desorption}

\section{Heterogeneous Catalysis}

\end{document}
