\documentclass[12pt, openany, letterpaper]{memoir}
\usepackage{NotesStyle}

\begin{document}
	\section*{Electrolyte Solutions Supplemental Notes}
	Consider finding the molal solubility of sparingly soluble salts using different levels of theory. First, find the molal solubility assuming an ideal solution. Then iteratively apply either the limiting Debye-H\"uckel equation or the Davies equation until the value converges.
	
	The limiting Debye-H\"uckel equation:
	\begin{equation*}
		\log\gamma_\pm = -0.5092\left|z_+z_-\right|\sqrt{I}
	\end{equation*}
	
	The Davies equation:
	\begin{equation*}
		\log\gamma_\pm = -0.5092\left|z_+z_-\right|\left[\dfrac{\left(\dfrac{I}{m^\circ}\right)^{\nicefrac{1}{2}}}{1+\left(\dfrac{I}{m^\circ}\right)^{\nicefrac{1}{2}}}-0.30\dfrac{I}{m^\circ}\right]
	\end{equation*}
	
	First, consider lead(II)chloride:
	\begin{equation*}
		\ch{PbCl2(s) -> Pb^{2+}(aq) + 2 Cl^-(aq)}\hspace{2em} K_{SP} = 1.6\times10^{-5}
	\end{equation*}
	
	\begin{tabular}{r|c|c|c|c}
		 & \multicolumn{2}{c}{Limiting Debye-H\"uckel Equation} & \multicolumn{2}{c}{Davies Equation} \\
		 Iteration & $\gamma_\pm$ & Molal Solubility & $\gamma_\pm$ & Molal Solubility  \\ \midrule
		0 & 1 & 0.0158740105197 & 1 & 0.0158740105197 \\
		1 & 0.599459625205 & 0.0264805332206 & 0.679391387341 & 0.0233650452677 \\
		2 & 0.516368839492 & 0.0307416120138 & 0.643029347667 & 0.0246862924333 \\
		3 & 0.490599120588 & 0.0323563778522 & 0.637923784628 & 0.024883866854 \\
		4 & 0.48162399219 & 0.0329593433406 & 0.637186445426 & 0.0249126619589 \\
		5 & 0.478371580596 & 0.0331834313817 & 0.637079529755 & 0.0249168428403 \\
		6 & 0.477175997129 & 0.0332665737908 & 0.637064017716 & 0.0249174495471 \\
		7 & 0.476734190378 & 0.0332974031233 & 0.637061766935 & 0.024917537582 \\
	\end{tabular}

	Note that the Davies equation converges much faster than the Limiting Debye-H\"uckel Equation
	
	Now for a more soluble salt like \ch{Ba(OH)2} with $K_{SP} = 5.0\times10^{-3}$ we get:
	
	\begin{tabular}{r|c|c|c|c}
		& \multicolumn{2}{c}{Limiting Debye-H\"uckel Equation} & \multicolumn{2}{c}{Davies Equation} \\
		Iteration & $\gamma_\pm$ & Molal Solubility & $\gamma_\pm$ & Molal Solubility  \\ \midrule
		0 & 1 & 0.107721734502 & 1 & 0.107721734502 \\
		1 & 0.263671886095 & 0.408544635141 & 0.536569564057 & 0.200760053714 \\
		2 & 0.0745669105917 & 1.44463185677 & 0.548290082776 & 0.196468507977 \\
		3 & 0.00758392135658 & 14.2039624934 & 0.54674153523 & 0.197024969863 \\
		4 & 2.24962815885e-07 & 478842.399256 & 0.546938174959 & 0.196954133819 \\
		5 & 0.0 & $\infty$ & 0.546913073963 & 0.19696317318 \\
	\end{tabular}

	Here the limiting Debye-H\"uckel equation diverges, as we saw in class, while the Davies equation converges
	
	To consider salt buffers, let's return to the lead(II) chloride problem. If we have a solution which is already 0.5 molal in some inert salt, such as \ch{NaCl}, then the ionic strength will be $0.5 + 3 * m$ where $m$ is the molal solubility. Here are the numbers for lead(II)chloride solubility in a 0.5 molal \ch{NaCl} solution
	
	\begin{tabular}{r|c|c|c|c}
		& \multicolumn{2}{c}{Limiting Debye-H\"uckel Equation} & \multicolumn{2}{c}{Davies Equation} \\
		Iteration & $\gamma_\pm$ & Molal Solubility & $\gamma_\pm$ & Molal Solubility  \\ \midrule
		0 & 1 & 0.0158740105197 & 1 & 0.0158740105197 \\
		1 & 0.176347310295 & 0.0900156089316 & 0.542240350477 & 0.0292748603193 \\
		2 & 0.127741041934 & 0.12426711321 & 0.546556466294 & 0.0290436789218 \\
		3 & 0.11183342862 & 0.141943341231 & 0.546475633368 & 0.0290479749698 \\
		4 & 0.104735737122 & 0.151562503458 & 0.54647713351 & 0.0290478952298 \\
		5 & 0.101144108273 & 0.1569444903 & 0.546477105665 & 0.0290478967099 \\
		6 & 0.0992113816142 & 0.160001909674 & 0.546477106182 & 0.0290478966824 \\
		7 & 0.0981369347281 & 0.16175368187 & 0.546477106172 & 0.0290478966829 \\
	\end{tabular}

	Since the ionic strength is mostly controlled by the \ch{NaCl}, the Davies equation  converges very quickly (though the ideal solution still gives very poor results). Note also that the molal solubility is greater than in the case of lead(II)chloride in pure water above. The Limiting Debye-H\"uckel Equation would converge quicker as well in most cases.
\end{document}